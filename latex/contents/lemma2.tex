If $a=b=\dfrac{1}{2},\tau=O(\dfrac{1}{p^2}\sqrt{\dfrac{s_0lnp}{n}}),\ s_0lnp=O(n)\ and\ \zeta >3, we\ have, for\ some\ constant\\ C>0,$ \\
\begin{align*}
    \pi^u(\Sigma \in \mathcal{U}(\zeta))>\left\{\dfrac{\lambda \zeta}{8}exp(-\dfrac{\lambda \zeta}{4}-C) \right\}^p
\end{align*}
\noindent
\begin{proof}
\noindent
이 Lemma는 논문의 Theorem4 의 증명에서 활용되는 Lemma이다.\\
\\
Gershgorin circle Thm에 의해, covariance matrix의 eigenvalue 들은 적어도\\
\\
$[\sigma_{jj}-\sum\limits_{k\neq j}|\sigma_{kj}|,\sigma_{jj}+\sum\limits_{k\neq j}|\sigma_{kj}|], j \in \{1,2,\cdots,p\}$ 안에 있다는 것을 알 \\
\\수 있다. 따라서,\\
\\
$\pi^u(\Sigma \in \mathcal{U}(\zeta))$ 
$\geq \pi^u(min_j(\sigma_{jj}-\sum_{k\neq j}|\sigma_{kj}|)>0,\zeta^{-1} \leq \lambda_{min}(\Sigma)\leq \lambda_{max}(\Sigma)\leq \zeta)$ 임을 보이면 된다.\\
먼저 $\min\limits_{j}(\sigma_{jj}-\sum\limits_{k\neq j}|\sigma_{kj}|)>0$을 살펴보면, 1-norm의 정의에 의해,\\
$||\Sigma||_1 = \max\limits_{1\leq j\leq n} \sum\limits_{i=1}^{m}|a_{ij}|$\ 이므로,\\
\\
$\lambda_{max}(\Sigma) \leq ||\Sigma||_1 = \max\limits_{j}(\sigma_{jj}+\sum\limits_{k\neq j}|\sigma_{kj}|)\leq \max\limits_{j}2\sigma_{jj}$\ 로 표현할 수 있다. 또한, G.C Thm에 의해, $\lambda_{min}(\Sigma)\geq \min\limits_{j}(\sigma_{jj}-\sum\limits_{k\neq j}|\sigma_{kj}|)$로 표현할 수 있다. 따라서,\\
이를 위의 식 $\zeta^{-1} \leq \lambda_{min}(\Sigma)\leq \lambda_{max}(\Sigma)\leq \zeta$에 적용해서 다시 표현하면,\\
\\
$\pi^u(\Sigma \in \mathcal{U}(\zeta)) \geq \pi^u(\zeta^{-1}\leq\min\limits_{j}(\sigma_{jj}-\sum\limits_{k\neq j}|\sigma_{kj}|)\leq 2\max\limits_{j}\sigma_{jj}\leq\zeta)$ 이고, $P(A)\geq P(A \cap B)=P(A|B)P(B)$ 성질을 이용하면,\\
\\
$\pi^u(\zeta^{-1}\leq\min\limits_{j}(\sigma_{jj}-\sum\limits_{k\neq j}|\sigma_{kj}|)\leq 2\max\limits_{j}\sigma_{jj}\leq\zeta)$\\
\\
$\geq \pi^u(\zeta^{-1}\leq\min\limits_{j}(\sigma_{jj}-\sum\limits_{k\neq j}|\sigma_{kj}|)\leq 2\max\limits_{j}\sigma_{jj}\leq\zeta \ | \max\limits_{k \neq j}|\sigma_{kj}|< (\zeta p)^{-1} )\pi^u(\max\limits_{k \neq j}|\sigma_{kj}|< (\zeta p)^{-1})$ 인데, 조건부에 의해 다음 부등식이 되고,\\
\\
$\geq \pi^u(\zeta^{-1}\leq\min\limits_{j}(\sigma_{jj}-\zeta^{-1})\leq 2\max\limits_{j}\sigma_{jj}\leq\zeta \ | \max\limits_{k \neq j}|\sigma_{kj}|< (\zeta p)^{-1} )\pi^u(\max\limits_{k \neq j}|\sigma_{kj}|< (\zeta p)^{-1})$ 에서, $\sigma_{jj}$ 항과 $\sigma_{ij}$는 독립이므로 조건부 항을 없앨 수 있다. 따라서,\\
\\
$=\pi^u(\zeta^{-1}\leq\min\limits_{j}(\sigma_{jj}-\zeta^{-1})\leq 2\max\limits_{j}\sigma_{jj}\leq\zeta)\pi^u(\max\limits_{k \neq j}|\sigma_{kj}|< (\zeta p)^{-1})$ 이다.\\
다시 정리해보면, 다음과 같은 부등식을 얻을 수 있다.\\
\\
$\pi^u(\Sigma \in \mathcal{U}(\zeta)) \geq \underline{\pi^u(\zeta^{-1}\leq\min\limits_{j}(\sigma_{jj}-\zeta^{-1})\leq 2\max\limits_{j}\sigma_{jj}\leq\zeta)} \ * \ \underline{\pi^u(\max\limits_{k \neq j}|\sigma_{kj}|< (\zeta p)^{-1})}$\\
\\
\\
먼저 첫 번째 밑줄 확률을 계산해보면,\\
\\
$\pi^u(\zeta^{-1}\leq\min\limits_{j}(\sigma_{jj}-\zeta^{-1})\leq 2\max\limits_{j}\sigma_{jj}\leq\zeta)$ $\geq \pi^u(2\zeta^{-1} \leq \sigma_{jj} \leq \dfrac{\zeta}{2}, \forall j)$\\
$\geq \prod\limits_{j=1}^p \pi^u(2\zeta^{-1}\leq \sigma_{jj} \leq \dfrac{\zeta}{2})$ 에서,\\
$\sigma_{jj}$가 $\Gamma(1,\dfrac{\lambda}{2})$ 를 따른다는 가정에 의해, $f(\sigma_{jj})=\dfrac{\lambda}{2}exp(-\dfrac{\lambda}{2}\sigma_{jj})$ 의 pdf 를 갖고, 가로 길이가 $\left(\dfrac{2}{\zeta},\dfrac{\zeta}{2}\right)$이고 세로 길이가 $\left(0,f\left(\dfrac{2}{\zeta}\right)\right)$ 인 직사각형을 생각하면, \\
\\
이는 pdf 의 전체 넓이 보다는 작으므로, 이를 통해 $\prod\limits_{j=1}^p\pi^u(2\zeta^{-1}\leq \sigma_{jj} \leq \dfrac{\zeta}{2})$\\
$=\left\{\left(\dfrac{\zeta}{2}-\dfrac{2}{\zeta}\right)\dfrac{\lambda}{2}exp(-\dfrac{\lambda \zeta}{4})\right\}^p \geq \left\{ \dfrac{\lambda \zeta}{8} exp(-\dfrac{\lambda \zeta}{4})\right\}^p$ 임을 알 수 있다.\\
\\
\\
이제 두 번째 밑줄 확률인 $\pi^u(\max\limits_{k \neq j}|\sigma_{kj}|< (\zeta p)^{-1})$를 계산할 건데, 먼저 이를 위한 lemma 하나를 소개하자.\\
\\
\ \\
\noindent
[lemma 1 in [12]]\\
\\
The univariate horseshoe density $p(\theta)$ satisfies the following:
\begin{align*}
    &(a) \lim\limits_{\theta \rightarrow 0}p(\theta) = \infty\\
    &(b) \ For\ \theta \neq 0, \dfrac{K}{2}log\left(1+\dfrac{4}{\theta^2}\right)<p(\theta)<Klog\left(1+\dfrac{2}{\theta^2}\right), \ where\ K=\dfrac{1}{\sqrt{2\pi^3}}
\end{align*} 

\noindent
따라서, 위의 lemma\ (b) 를 통해, $\pi^u(\max\limits_{k \neq j}|\sigma_{kj}|< (\zeta p)^{-1})$의 계산을 위한 부등식인 $\pi^u(\sigma_{kj})\leq \dfrac{1}{\tau \sqrt{2\pi^3}}ln\left(1+\dfrac{2\tau^2}{\sigma_{kj}^2}\right)$ 를 알 수 있다.\\
\\
한편, $|\sigma_{kj}|$ 는 이대일 변환이므로, 2가 곱해져서,\\
\\
$\pi^u(|\sigma_{kj}|\geq(\zeta p)^{-1})\leq \dfrac{1}{\zeta}\sqrt{\dfrac{2}{\pi^3}} \displaystyle \int_{(\zeta p)^{-1}}^{\infty}ln\left(1+\dfrac{2\tau^2}{x^2}\right) dx$ 가 되고,\\
$\leq \sqrt{\dfrac{2}{\pi^3}} \displaystyle \int_{(\zeta p)^{-1}}^{\infty}\dfrac{2\tau^2}{x^2} dx$ $( \because ln(1+x) \leq x, \ when\ x\geq 0)$\\
\\
$= \sqrt{\dfrac{2}{\pi^3}} \displaystyle \int_{(\zeta p)^{-1}}^{\infty}\dfrac{2\tau}{x^2} dx$
$=\dfrac{2\sqrt{2}}{\sqrt{\pi^3}}\tau\zeta p $ 임을 알 수 있다.\\
\\
\\
이를 통해 이제 $\pi^u(\max\limits_{k \neq j}|\sigma_{kj}|< (\zeta p)^{-1})$ 를 계산해보면,\\
\\
$\pi^u(\max\limits_{k \neq j}|\sigma_{kj}|< (\zeta p)^{-1}) = \prod\limits_{k\neq j} \left\{    1-\pi^u(|\sigma_{kj}|\geq (\zeta p)^{-1})   \right\} = \left(1-\dfrac{2\sqrt{2}}{\sqrt{\pi^3}}\tau\zeta p\right)^{p(p-1)}$ \\
$\geq \left(1-\dfrac{2\sqrt{2}}{\sqrt{\pi^3}}\tau\zeta p\right)^{p^2} \ \cdots \ (1) $ $(\because$ 괄호안의 값은확률로 1보다 작으므로)\\
\\
\\
$\geq exp\left(-\dfrac{4\sqrt{2}}{\sqrt{\pi^3}}\tau \zeta p^3\right) \ (\because log(1-x) \geq -2x,\ when\ x\leq \dfrac{1}{2})$ 이다.\\
한편, 주어진 조건에서 $\tau=O\left(\dfrac{1}{p^2}\sqrt{\dfrac{s_0lnp}{n}}\right), s_0lnp=O(n)$ 이라 했으므로,\\
$\dfrac{\tau}{\dfrac{1}{p^2}\sqrt{\dfrac{s_olnp}{n}}}\leq c_1, \dfrac{s_olnp}{n}\leq c_2,\ c_1,c_2>0$ 임을 알 수 있다. 이들을 조합하면\\
\\
$\Rightarrow \ \tau p^2 \leq \sqrt{c_2}c_1 \defeq c_3 \ \Rightarrow \tau p^3 \leq c_3p\ \Rightarrow \ -\tau p^3 \geq -c_3p $\ 이다. 이를 (1) 식에서 활용하면, $exp\left(-\dfrac{4\sqrt{2}}{\sqrt{\pi^3}}\tau \zeta p^3\right) \geq exp\left(-c_3 \dfrac{4\sqrt{2}}{\sqrt{\pi^3}} \zeta p\right)=exp(-Cp)$ 이다. 따라서, 두번째 밑줄 확률의 부등식을 다음과 같이 구할 수 있다.\\
\\
$\pi^u(\max\limits_{k \neq j}|\sigma_{kj}|< (\zeta p)^{-1}) \geq exp(-Cp) $ \\
\\
이제 첫 번째 밑줄 식과 두 번째 밑줄 식의 결과를 종합하면,\\
\\
$\pi^u(\Sigma \in \mathcal{U}(\zeta)) \geq \pi^u(\zeta^{-1}\leq\min\limits_{j}(\sigma_{jj}-\zeta^{-1})\leq 2\max\limits_{j}\sigma_{jj}\leq\zeta) \ * \ \pi^u(\max\limits_{k \neq j}|\sigma_{kj}|< (\zeta p)^{-1})$\\
\\
$\geq \left\{ \dfrac{\lambda \zeta}{8} exp(-\dfrac{\lambda \zeta}{4})\right\}^p exp(-Cp)=\left\{ \dfrac{\lambda \zeta}{8} exp(-\dfrac{\lambda \zeta}{4}-C)\right\}^p$ 임을 알 수 있다.\\








\end{proof}
