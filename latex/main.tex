\documentclass[11pt]{article}
\usepackage{subcaption}
\usepackage{arydshln}
\usepackage{booktabs}
\usepackage{xcolor}
\usepackage{multirow}
\usepackage{caption}
\usepackage{kotex}
\usepackage{physics}
\usepackage{bibentry}
%\usepackage{mathtools}
\usepackage[english]{babel}
\usepackage{amsthm}
\usepackage{amssymb}
% \usepackage{enumitem}
\usepackage{enumerate}
\usepackage{hyperref}
\newcommand{\subscript}[2]{$#1_#2$}

\bibliographystyle{apalike}
\usepackage[sort&compress]{natbib}

\newtheorem{theorem}{Theorem}[section]
\newtheorem*{theorem*}{Theorem}
\newtheorem{lemma}[theorem]{Lemma}
\newtheorem*{lemma*}{Lemma}
\newtheorem{definition}{Definition}[section]
\newtheorem{remark}{Remark}[section]

 % Top and bottom rules for table


% There are many different themes available for Beamer. A comprehensive
% list with examples is given here:
% http://deic.uab.es/~iblanes/beamer_gallery/index_by_theme.html
% You can uncomment the themes below if you would like to use a different
% one:
%\usetheme{AnnArbor}
%\usetheme{Antibes}
%\usetheme{Bergen}
%\usetheme{Berkeley}
%\usetheme{Berlin}
%\usetheme{Boadilla}
%\usetheme{boxes}
%\usetheme{CambridgeUS}
%\usetheme{Copenhagen}
%\usetheme{Darmstadt}
%\usetheme{default}
%\usetheme{Frankfurt}
%\usetheme{Goettingen}
%\usetheme{Hannover}
%\usetheme{Ilmenau}
%\usetheme{JuanLesPins}
%\usetheme{Luebeck}
%\usetheme{Madrid}
%\usetheme{Malmoe}
%\usetheme{Marburg}
%\usetheme{Montpellier}
%\usetheme{PaloAlto}
%\usetheme{Pittsburgh}
%\usetheme{Rochester}
%\usetheme{Singapore}
%\usetheme{Szeged}
%\usetheme{Warsaw}
%\usecolortheme{rose}
%%%%%%%%%%%%%%%%%%%%%%%%%%%%%%%%%%%%%%%%%%%%%%%%%%%%
%%         문자에 액센트
%%%%%%%%%%%%%%%%%%%%%%%%%%%%%%%%%%%%%%%%%%%%%%%%%%%%

% arg
\DeclareMathOperator*{\argmax}{arg  max}
\DeclareMathOperator*{\argmin}{arg  min}
\newcommand{\defeq}{\mathrel{\mathop:}=}

% true
\newcommand{\thetat}{{\theta^t}}
\newcommand{\pt}{{p^t}}
\newcommand{\Et}{{\mathbb{E}^t}}
\newcommand{\Vt}{{\mathbb{V}ar^t}}


% hat
\newcommand{\dhat}{\hat{d}}
\newcommand{\fhat}{\hat{f}}
\newcommand{\ghat}{\hat{g}}
\newcommand{\hhat}{\hat{h}}
\newcommand{\mhat}{\hat{m}}
\newcommand{\phat}{\hat{p}}
\newcommand{\uhat}{\hat{u}}
\newcommand{\vhat}{\hat{v}}
\newcommand{\xhat}{\hat{x}}
\newcommand{\yhat}{\hat{y}}
\newcommand{\Fhat}{\hat{F}}
\newcommand{\Ghat}{\hat{G}}
\newcommand{\Hhat}{\hat{H}}
\newcommand{\Ihat}{\hat{I}}
\newcommand{\Vhat}{\hat{V}}
\newcommand{\Xhat}{\hat{X}}
\newcommand{\Yhat}{\hat{Y}}
\newcommand{\alphahat}{{\hat{\alpha}}}
\newcommand{\betahat}{{\hat{\beta}}}
\newcommand{\gammahat}{{\hat{\gamma}}}
\newcommand{\etahat}{{\hat{\eta}}}
\newcommand{\muhat}{{\hat{\mu}}}
\newcommand{\pihat}{{\hat{\pi}}}
\newcommand{\phihat}{{\hat{\phi}}}
\newcommand{\psihat}{{\hat{\psi}}}
\newcommand{\sigmahat}{{\hat{\sigma}}}
\newcommand{\thetahat}{{\hat{\theta}}}
\newcommand{\Sigmahat}{{\hat{\Sigma}}}

% bar
\newcommand{\pbar}{\bar{p}}
\newcommand{\xbar}{\bar{x}}
\newcommand{\ybar}{\bar{y}}
\newcommand{\zbar}{\bar{z}}
\newcommand{\Dbar}{\bar{D}}
\newcommand{\Tbar}{\bar{X}}
\newcommand{\Xbar}{\bar{X}}
\newcommand{\Ybar}{\bar{Y}}
\newcommand{\Zbar}{\bar{Z}}
\newcommand{\thetabar}{\bar{\theta}}
\newcommand{\etabar}{\bar{\eta}}

% tilde
\newcommand{\ptilde}{ \tilde{p}}
\newcommand{\Rtilde}{ \tilde{R}}
\newcommand{\Vtilde}{ \tilde{V}}
\newcommand{\Xtilde}{ \tilde{X}}
\newcommand{\Ytilde}{ \tilde{Y}}
\newcommand{\betatilde}{ \tilde{\beta}}
\newcommand{\mutilde}{ \tilde{\mu}}
\newcommand{\xitilde}{ \tilde{\xi}}
\newcommand{\thetatilde}{ \tilde{\theta}}
\newcommand{\sigmatilde}{ \tilde{\sigma}}
\newcommand{\Sigmatilde}{ \tilde{\Sigma}}
\newcommand{\Vartilde}{ \tilde{Var}}


% dot
\newcommand{\Edot}{ \dot{E}}
\newcommand{\Idot}{ \dot{I}}
\newcommand{\Rdot}{ \dot{R}}
\newcommand{\Sdot}{ \dot{S}}
\newcommand{\fdot}{ \dot{f}}
\newcommand{\xdot}{ \dot{x}}


% bold
\newcommand{\bfc}{\mathbf{c}}
\newcommand{\bff}{ \mathbf{f}}
\newcommand{\bfm}{ \mathbf{m}}
\newcommand{\bfn}{ \mathbf{n}}
\newcommand{\bfr}{ \mathbf{r}}
\newcommand{\bfs}{ \mathbf{s}}
\newcommand{\bft}{ \mathbf{t}}
\newcommand{\bfu}{ \mathbf{u}}
\newcommand{\bfw}{ \mathbf{w}}
\newcommand{\bfx}{ \mathbf{x}}
\newcommand{\bfy}{ \mathbf{y}}
\newcommand{\bfz}{ \mathbf{z}}
\newcommand{\bfE}{ \mathbf{E}}
\newcommand{\bfF}{ \mathbf{F}}
\newcommand{\bfO}{ \mathbf{O}}
\newcommand{\bfX}{ \mathbf{X}}
\newcommand{\bfY}{ \mathbf{Y}}
\newcommand{\bfZ}{\mathbf{Z}}
\newcommand{\bfzero}{{\bf 0}}
\newcommand{\bfone}{{\bf 1}}


% blackboard bold
\newcommand{\bbC}{{ \mathbb{C}}}
\newcommand{\bbE}{{ \mathbb{E}}}
\newcommand{\bbN}{ \mathbb{N}}
\newcommand{\bbP}{ \mathbb{P}}
\newcommand{\bbR}{ \mathbb{R}}
\newcommand{\bbX}{ \mathbb{X}}
\newcommand{\bbY}{ \mathbb{Y}}
\newcommand{\bbZ}{ \mathbb{Z}}

% calligraphy
\newcommand{\calA}{\mathcal{A}}
\newcommand{\calB}{\mathcal{B}}
\newcommand{\calC}{\mathcal{C}}
\newcommand{\calD}{\mathcal{D}}
\newcommand{\calE}{\mathcal{E}}
\newcommand{\calF}{\mathcal{F}}
\newcommand{\calG}{\mathcal{G}}
\newcommand{\calH}{\mathcal{H}}
\newcommand{\calL}{\mathcal{L}}
\newcommand{\calM}{\mathcal{M}}
\newcommand{\calP}{\mathcal{P}}
\newcommand{\calQ}{\mathcal{Q}}
\newcommand{\calS}{\mathcal{S}}
\newcommand{\calT}{\mathcal{T}}
\newcommand{\calX}{ \mathcal{X}}
\newcommand{\calY}{\mathcal{Y}}
\newcommand{\calZ}{\mathcal{Z}}


% bold symbol
\newcommand{\balpha}{ \boldsymbol{\alpha}}
\newcommand{\bbeta}{ \boldsymbol{\beta}}
\newcommand{\bepsilon}{ \boldsymbol{\epsilon}}
\newcommand{\blambda}{ \boldsymbol{\lambda}}
\newcommand{\bmu}{ \boldsymbol{\mu}}
\newcommand{\bnu}{ \boldsymbol{\nu}}
\newcommand{\bpi}{ \boldsymbol{\pi}}
\newcommand{\bphi}{ \boldsymbol{\phi}}
\newcommand{\bpsi}{ \boldsymbol{\psi}}
\newcommand{\btheta}{ \boldsymbol{\theta}}
\newcommand{\bomega}{ \boldsymbol{\omega}}
\newcommand{\bxi}{ \boldsymbol{\xi}}

\newcommand{\indep}{\perp \!\!\! \perp}

\newcommand{\bed}{\begin{itemize}}
   \newcommand{\eed}{\end{itemize}}
\newcommand{\vs}{\vspace}


\newcommand{\jsum}{\sum_{j=1}^{p}}


%\newcommand{\bfc}{\mathbf{c}}
%\newcommand{\bff}{ \mathbf{f}}
\newcommand{\bfg}{ \mathbf{g}}
%\newcommand{\bfm}{ \mathbf{m}}
%\newcommand{\bfn}{ \mathbf{n}}
\newcommand{\bfp}{ \mathbf{p}}
%\newcommand{\bfr}{ \mathbf{r}}
%\newcommand{\bfs}{ \mathbf{s}}
%\newcommand{\bft}{ \mathbf{t}}
%\newcommand{\bfu}{ \mathbf{u}}
\newcommand{\bfv}{ \mathbf{v}}
%\newcommand{\bfw}{ \mathbf{w}}
%\newcommand{\bfx}{ \mathbf{x}}
%\newcommand{\bfy}{ \mathbf{y}}
%\newcommand{\bfz}{ \mathbf{z}}
%\newcommand{\bfN}{ \mathbf{N}}
\newcommand{\bfI}{ \mathbf{I}}
\newcommand{\bfJ}{ \mathbf{J}}
%\newcommand{\bfE}{ \mathbf{E}}
%\newcommand{\bfF}{ \mathbf{F}}
%\newcommand{\bfO}{ \mathbf{O}}
%\newcommand{\bfS}{ \mathbf{S}}
\newcommand{\bfV}{ \mathbf{V}}
%\newcommand{\bfX}{ \mathbf{X}}
%\newcommand{\bfY}{ \mathbf{Y}}
%\newcommand{\bfZ}{\mathbf{Z}}
%\newcommand{\bfzero}{{\bf 0}}
%\newcommand{\bfone}{{\bf 1}}
%\newcommand{\balpha}{ \boldsymbol{\alpha}}
%\newcommand{\bbeta}{ \boldsymbol{\beta}}
%\newcommand{\bepsilon}{ \boldsymbol{\epsilon}}
\newcommand{\bvarepsilon}{ \boldsymbol{\varepsilon}}
%\newcommand{\blambda}{ \boldsymbol{\lambda}}
%\newcommand{\bmu}{ \boldsymbol{\mu}}
%\newcommand{\bnu}{ \boldsymbol{\nu}}
%\newcommand{\bpi}{ \boldsymbol{\pi}}
%\newcommand{\bphi}{ \boldsymbol{\phi}}
%\newcommand{\bpsi}{ \boldsymbol{\psi}}
%\newcommand{\btheta}{ \boldsymbol{\theta}}
%\newcommand{\bomega}{ \boldsymbol{\omega}}
%\newcommand{\bxi}{ \boldsymbol{\xi}}
\newcommand{\bfeta}{ \boldsymbol{\eta}}
\newcommand{\bsigma}{ \boldsymbol{\sigma}}
\newcommand{\bzero}{\boldsymbol{0}}
\newcommand{\bone}{\boldsymbol{1}}

%\usepackage{geometry}
% \geometry{
% a4paper,
% total={170mm,257mm},
% left=20mm,
% right=20mm,
% top=20mm,
% bottom = 33mm
% }


%\setlist[itemize,2]{label=$\triangleright$}
%\setlist[itemize,3]{label=$\centerdot$}

\usepackage[left=3cm,right=3cm,top=2cm,bottom=2cm,a4paper]{geometry} % 페이지 여백 옵션

%---------------------------------------------------------------------%

% Let's get started

% theorem, proof, lemma, definition


\begin{document}

\author{신창수, 오정훈, 장태영, 이경원, 정진욱, 김성민, 백승찬}
\date{\today}

\title{The beta - mixture shrinkage prior for sparse covariances with
near-minimax posterior convergence}

\maketitle

\section*{To Do List}

\begin{itemize}
    \item 소개 작성
    \item 스타일 통일
    \item 기호 통일 ($R$, $\mathbb{R}$)
    \item 백승찬 - Lemma 6 수정
    \item theorem, lemma 등 한글화 
    \item equation numbering
\end{itemize}

\section*{소개}

추가 예정

\section{신창수 - Theorem 3 and Lemma 2}

\subsection{Theorem 3}

(The upper bound of the packing number).\\
If ${\zeta}^4 \leq p, \ p \asymp n^{\beta} \ for\ some\ 0 \leq \beta \leq 1,\ s_n = c_1n \epsilon_n^2/lnp,\ L_n = c_2n \epsilon_n^2\ \ and\ \delta_n=\epsilon_n/\zeta^3\ for\ some\ constants\ c_1>1\ and c_2>0, we\ have$
\begin{align*}
    lnD(\epsilon_n,P_n,d) \ \leq \ (12+1/\beta)c_1n\epsilon_n^2
\end{align*}
\begin{proof}
이 정리는 Lemma1의 세 가지 조건 중 첫 번째 조건이 성립함을 보이는 정리이다. 증명에 앞서 각각의 용어들에 대해 정의하자.\\
\\
$D(\epsilon_n,P_n,d)$ 은 $P_n$ 안에서, 각각의 쌍이 이루는 거리들의 거리가 $\epsilon_n$보다 크거나 같은 점들의 최대 개수로 정의한다. 또한,\\
\\
$P_n=\{f_\Sigma \ :\ |s(\Sigma,\delta_n)|\leq s_n, \ \zeta^{-1} \leq \lambda_{min}(\Sigma) \leq \lambda_{max}(\Sigma) \leq \zeta,||\Sigma||_{max} \leq L_n  \}$\\
\\
$\mathcal{U}(\delta_n,s_n,L_n,\zeta)=\{\Sigma \in \mathcal{c}_p : |s(\Sigma,\delta_n)|\leq s_n,\zeta^{-1} \leq \lambda_{min}(\Sigma) \leq \lambda_{max}(\Sigma) \leq \zeta,||\Sigma||_{max} \leq L_n \} $\\ 
\\
$s_0$ : 공분산 행렬의 비대각 원소 중 0이 아닌 원소들의 상한\\
\\
$\epsilon_n \defeq \{\dfrac{(p+s_0)lnp}{n}\}^{\dfrac{1}{2}}$
\noindent
라고 정의하자.\\
\\
최종적으로는, $lnD(\epsilon_n,\mathcal{P}_n,d)\ \leq \ (12+1/\beta)c_1n\epsilon_n^2$ 임을 보일 건데, 그 전에 먼저 lemma3 소개하자.\\
\\
\noindent
[Lemma3](Lemma A.1 in [6])
\\
If  $P_{\Omega_{k}}$ is the density of $N_p(0,\Omega_{k}^{-1})$, k=1,2, then for all $\Omega_k \in \mu_{0}^+$, k=1,2, and $d_i$, i=1,2,…,p, eigenvalues of $A=\Sigma_1^{-1/2}\Sigma_2\Sigma_1^{-1/2}$, we have that for some $\delta > 0$ and constant $c_o>0$,\\
\\
(1)$c_0^{-1}||\Sigma_1-\Sigma_2||_2^2\leq \sum_{i=1}^{p}|d_i-1|^2\leq c_o||\Sigma_1-\Sigma_2||_2^2$\\
\\
(2)$h(p_{\Omega_{1}},p_{\Omega_{2}})<\delta$ implies $\max\limits_i|d_i-1|<1$ and $||\Omega_1-\Omega_2||_2 \leq c_oh^2(p_{\Omega_{1}},p_{\Omega_{2}})$\\
\\
(3) $h^2(p_{\Omega_{1}},p_{\Omega_{2}}) \leq c_o||\Omega_1-\Omega_2||_2^2$

\noindent
* 여기서, $h()$는 hellinger distance 로, 우리 논문에서 $d()$와 같음\\
\\
따라서, lemma3-(3)에 의해,
$d(f_{\Sigma_1},f_{\Sigma_2}) \leq c\zeta||\Omega_1-\Omega_2||_F$ 임을 알 수 있고, 여기서 $\Omega$는 $\Sigma^{-1}$ 이다.\\
\\
위 lemma3-(3)과 우리 논문의 lemma5를 통해, $\Omega_1=\Sigma_1^{-1}\Sigma_1\Sigma_1^{-1}$ 임을 이용\\
\\
하면,
$d(f_{\Sigma_1},f_{\Sigma_2}) \leq C\zeta^3||\Sigma_1-\Sigma_2||_F$ \ (1) 식을 얻을 수 있다.
\noindent
$\epsilon$-packing의\\
\\
정의에 의해, $d(f_{\Sigma_{i}},f_{\Sigma_{j}})\geq \epsilon_n$ 으로부터, (1) 식과 결합하여, $||\Sigma_1-\Sigma_2||_F \geq \dfrac{\epsilon_n}{C\zeta^3}$ 의\\
결과를 얻을 수 있다. 따라서, 집합을 $\mathcal{P}_n$ 에서, $\mathcal{U}(\delta_n,s_n,L_n,\zeta)$ 으로 바꾸고,\\
\\
거리를 Frobenius norm 으로 바꾸어서 위의 결과를 적용하자. 여기서 격자\\
\\
개념을 생각해보면, 격자의 대각선 부분들까지 고려해주어야 한다. 따라서,\\
\\
$lnD(\epsilon_n,P_n,d) \leq lnD(\dfrac{\epsilon_n}{C\zeta^3},\mathcal{U}(\delta_n,s_n,L_n,\zeta),||\cdot||_F)\\
\\
\leq ln\left\{\left(\dfrac{L_n\sqrt{p+j}}{\dfrac{\epsilon_n}{C\zeta^3}}\right)^p \displaystyle \sum\limits_{j=1}^{s_n}\left(\dfrac{\sqrt{p+j}\dfrac{1}{\sqrt{2}}L_n}{\dfrac{\epsilon_n}{C\zeta^3}}\right)^{j}{\dfrac{p}{2} \displaystyle \choose j} \right\}$\\
여기서, $\sqrt{p+j}$는 격자의 대각선을 고려해준 항이고, $\dfrac{1}{\sqrt{2}}$는 Frobenius norm에서, symmetric term들의 중복을 고려해 준 값이다.\\
\\
한편, $\left(\dfrac{L_n\sqrt{p+j}}{\dfrac{\epsilon_n}{C\zeta^3}}\right)$에서 $j\leq s_n \leq p^2$임을 통해, j를 p에 대한 부등식으로 적절히 바꾸어주면,
$\dfrac{L_n\sqrt{p+j}}{\dfrac{\epsilon_n}{C\zeta^3}}\leq \dfrac{2p\zeta^3L_n}{\epsilon_n}$ 을 얻을 수 있다.\\
\\ 
따라서, $ln\left\{\left(\dfrac{L_n\sqrt{p+j}}{\dfrac{\epsilon_n}{C\zeta^3}}\right)^p \displaystyle \sum\limits_{j=1}^{s_n}\left(\dfrac{\sqrt{p+j}\dfrac{1}{\sqrt{2}}L_n}{\dfrac{\epsilon_n}{C\zeta^3}}\right)^{j}{\dfrac{p}{2} \choose j} \right \}$\\
\\
$=ln\left\{\left(\dfrac{2pC\zeta^3L_n}{\epsilon_n}\right)^p \displaystyle \sum\limits_{j=1}^{s_n}\left(\dfrac{\sqrt{2}C\zeta^3L_np}{\epsilon_n}\right)^j{\dfrac{p}{2} \choose j}\right\}$\\
$= ln\left[((2p)^p(\sqrt{2}p)^{s_n})\left(\dfrac{C\zeta^3L_n}{\epsilon_n}\right)^p \displaystyle \sum\limits_{j=1}^{s_n} \left(\dfrac{C\zeta^3L_n}{\epsilon_n}\right)^j{\dfrac{p}{2} \choose j}\right]$\\
$=pln2+plnp+s_n(\dfrac{1}{2}ln2+lnp)+pln\left(\dfrac{CL_n\zeta^3}{\epsilon_n}\right)+ln\left( \displaystyle \sum\limits_{j=1}^{s_{n}}\left(\dfrac{2CL_n\zeta^3}{\epsilon_n}\right)(\dfrac{p^2}{2})^j\right)$\\
$\leq pln2+plnp+s_n(\dfrac{1}{2}ln2+lnp)+pln\left(\dfrac{CL_n\zeta^3}{\epsilon_n}\right)+s_{n}ln\left(\dfrac{2CL_n\zeta^3p^2}{\epsilon_n}\right)$\\
$\leq pln2+plnp+\dfrac{1}{2}s_nln2+s_nlnp+(p+s_n)ln(2CL_n)+(p+s_n)ln\zeta^3+(p+s_n)ln\dfrac{1}{\epsilon_n}+2s_nlnp $ 에서 적절한 상수를 곱해주면,\\
$\leq 2(p+s_n)lnp+(p+s_n)ln(2CL_n)+(p+s_n)ln\zeta^3+(p+s_n)ln\dfrac{1}{\epsilon_n}+2s_nlnp $\\
\\
이다. 먼저, $(p+s_n)ln(2CL_n)\leq 6s_nlnp$ 임을 보일건데,\\
\\
위에서 정의한 $s_n,\epsilon_n,L_n$을 통해, $s_n=c_1(p+s_0)$ 이므로,\\
\\
$p+s_n=(1+c_1)p+c_1s_0$임을 알 수 있다. 또한, $2CL_n=2c_2n\epsilon_n^2$ 이다.\\
\\
이를 좌변에 대입하면,$(p+s_n)ln(2CL_n)=((c_1+1)+c_1s_0)ln2c_2n\epsilon_n^2$\\
\\
$=((c_1+1)p+c_1s_0)ln2c_2(p+s_0)lnp$이다. $c_1 > 1$ 가정에 의해,\\
\\
$((c_1+1)+c_1s_0)ln2c_2n\epsilon_n^2\leq 2c_1(p+s_0)ln(2c_2(p+s_0)lnp)$이다. 한편, $s_0$는\\
\\
비대각 원소 중 0이 아닌 것들의 개수의 상한이므로 $s_o\leq p^2$ 임을 알 수 있다. \\
\\
따라서, 적절한 차수 $p^3$을 통해 $2c_2(p+s_0)lnp<p^3$ 을 얻을 수 있다. 이를 통해,\\
\\
$(p+s_n)ln(2CL_n)\leq 2c_1(p+s_0)lnp^3=6s_nlnp$ 부등식을 얻을 수 있다.\\
\\
이를 정리하면,\\
$2(p+s_n)lnp+(p+s_n)ln(2CL_n)+(p+s_n)ln\zeta^3+(p+s_n)ln\dfrac{1}{\epsilon_n}+2s_nlnp $\\
$\leq 2(p+s_n)lnp+6s_nlnp+(p+s_0)ln\zeta^3+(p+s_n)ln(\dfrac{1}{\epsilon_n})+2s_nlnp $이다.\\
이제 $(p+s_0)ln\zeta^3\leq \dfrac{3}{4}(p+s_n)lnp$ 임을 보이자.\\
$(p+s_n)ln\zeta^3=\dfrac{3}{4}(p+s_n)ln\zeta^4$인데, 가정에 의해, $\zeta^4\leq p$이므로,\\
\\
$(p+s_n)ln\zeta^3\leq \dfrac{3}{4}(p+s_n)lnp$임을 알 수 있다. 이를 대입하여 정리하면,\\
$\leq 2(p+s_n)lnp+6s_nlnp+(p+s_0)ln\zeta^3+(p+s_n)ln(\dfrac{1}{\epsilon_n})+2s_nlnp $\\
$\leq 2(p+s_n)lnp+6s_nlnp+\dfrac{3}{4}(p+s_n)lnp+(p+s_n)ln(\dfrac{1}{\epsilon_n})+2s_nlnp $이다.\\
\\
\noindent
세번째 항에, 앞에서 정의한\ $\epsilon_n$을 대입하여 정리하면,\\
\\
$(p+s_n)ln(\dfrac{1}{\epsilon_n})=\dfrac{1}{2}(p+s_n)ln\dfrac{n}{(p+s_o)lnp}$임을 알 수 있고,\\
\\
우변$= \dfrac{1}{2\beta}(p+s_n)ln\left(\dfrac{n}{(p+s_o)lnp}\right)^\beta$\\
\\
$=\dfrac{1}{2\beta}(p+s_n)lnn^\beta-\dfrac{1}{2}(p+s_n)ln(p+s_0)lnp$\\
$\leq \dfrac{1}{2\beta}(p+s_n)lnn^\beta=\dfrac{1}{2\beta}(p+s_n)lnp$ 이다.($\because p\asymp n^\beta  $)\\
\\
이를 다시 처음 부등식에서 정리하면,\\
\\
$\leq 2(p+s_n)lnp+6s_nlnp+\dfrac{3}{4}(p+s_n)lnp+\dfrac{1}{2\beta}(p+s_n)lnp+2s_nlnp $이다.\\
\\
이 부등식에 적절한 상수배를 해주면,\\
\\
$\leq 6s_nlnp+\dfrac{11}{4}(p+s_n)lnp+\dfrac{1}{2\beta}(p+s_n)lnp+2s_nlnp$\\
$(6+\dfrac{1}{2\beta})(\dfrac{s_n}{c_1}+s_n)lnp< \dfrac{1}{2}(12+\dfrac{1}{\beta})(c_1+c_1)n\epsilon_n^2$\ ($\because c_1>1$)\\
\\
$=(12+\dfrac{1}{2\beta})c_1n\epsilon_n^2$ 이다.\\
따라서, $lnD(\epsilon_n,P_n,d)\leq(12+\dfrac{1}{\beta})c_1n\epsilon_n^2$이 성립해서,\\
\\
이를 통해 Lemma1의 첫 번째 조건이 성립함을 알 수 있다.


\end{proof}



\subsection{Lemma 2}

If $a=b=\dfrac{1}{2},\tau=O(\dfrac{1}{p^2}\sqrt{\dfrac{s_0lnp}{n}}),\ s_0lnp=O(n)\ and\ \zeta >3, we\ have, for\ some\ constant\\ C>0,$ \\
\begin{align*}
    \pi^u(\Sigma \in \mathcal{U}(\zeta))>\left\{\dfrac{\lambda \zeta}{8}exp(-\dfrac{\lambda \zeta}{4}-C) \right\}^p
\end{align*}
\noindent
\begin{proof}
\noindent
이 Lemma는 논문의 Theorem4 의 증명에서 활용되는 Lemma이다.\\
\\
Gershgorin circle Thm에 의해, covariance matrix의 eigenvalue 들은 적어도\\
\\
$[\sigma_{jj}-\sum\limits_{k\neq j}|\sigma_{kj}|,\sigma_{jj}+\sum\limits_{k\neq j}|\sigma_{kj}|], j \in \{1,2,\cdots,p\}$ 안에 있다는 것을 알 \\
\\수 있다. 따라서,\\
\\
$\pi^u(\Sigma \in \mathcal{U}(\zeta))$ 
$\geq \pi^u(min_j(\sigma_{jj}-\sum_{k\neq j}|\sigma_{kj}|)>0,\zeta^{-1} \leq \lambda_{min}(\Sigma)\leq \lambda_{max}(\Sigma)\leq \zeta)$ 임을 보이면 된다.\\
먼저 $\min\limits_{j}(\sigma_{jj}-\sum\limits_{k\neq j}|\sigma_{kj}|)>0$을 살펴보면, 1-norm의 정의에 의해,\\
$||\Sigma||_1 = \max\limits_{1\leq j\leq n} \sum\limits_{i=1}^{m}|a_{ij}|$\ 이므로,\\
\\
$\lambda_{max}(\Sigma) \leq ||\Sigma||_1 = \max\limits_{j}(\sigma_{jj}+\sum\limits_{k\neq j}|\sigma_{kj}|)\leq \max\limits_{j}2\sigma_{jj}$\ 로 표현할 수 있다. 또한, G.C Thm에 의해, $\lambda_{min}(\Sigma)\geq \min\limits_{j}(\sigma_{jj}-\sum\limits_{k\neq j}|\sigma_{kj}|)$로 표현할 수 있다. 따라서,\\
이를 위의 식 $\zeta^{-1} \leq \lambda_{min}(\Sigma)\leq \lambda_{max}(\Sigma)\leq \zeta$에 적용해서 다시 표현하면,\\
\\
$\pi^u(\Sigma \in \mathcal{U}(\zeta)) \geq \pi^u(\zeta^{-1}\leq\min\limits_{j}(\sigma_{jj}-\sum\limits_{k\neq j}|\sigma_{kj}|)\leq 2\max\limits_{j}\sigma_{jj}\leq\zeta)$ 이고, $P(A)\geq P(A \cap B)=P(A|B)P(B)$ 성질을 이용하면,\\
\\
$\pi^u(\zeta^{-1}\leq\min\limits_{j}(\sigma_{jj}-\sum\limits_{k\neq j}|\sigma_{kj}|)\leq 2\max\limits_{j}\sigma_{jj}\leq\zeta)$\\
\\
$\geq \pi^u(\zeta^{-1}\leq\min\limits_{j}(\sigma_{jj}-\sum\limits_{k\neq j}|\sigma_{kj}|)\leq 2\max\limits_{j}\sigma_{jj}\leq\zeta \ | \max\limits_{k \neq j}|\sigma_{kj}|< (\zeta p)^{-1} )\pi^u(\max\limits_{k \neq j}|\sigma_{kj}|< (\zeta p)^{-1})$ 인데, 조건부에 의해 다음 부등식이 되고,\\
\\
$\geq \pi^u(\zeta^{-1}\leq\min\limits_{j}(\sigma_{jj}-\zeta^{-1})\leq 2\max\limits_{j}\sigma_{jj}\leq\zeta \ | \max\limits_{k \neq j}|\sigma_{kj}|< (\zeta p)^{-1} )\pi^u(\max\limits_{k \neq j}|\sigma_{kj}|< (\zeta p)^{-1})$ 에서, $\sigma_{jj}$ 항과 $\sigma_{ij}$는 독립이므로 조건부 항을 없앨 수 있다. 따라서,\\
\\
$=\pi^u(\zeta^{-1}\leq\min\limits_{j}(\sigma_{jj}-\zeta^{-1})\leq 2\max\limits_{j}\sigma_{jj}\leq\zeta)\pi^u(\max\limits_{k \neq j}|\sigma_{kj}|< (\zeta p)^{-1})$ 이다.\\
다시 정리해보면, 다음과 같은 부등식을 얻을 수 있다.\\
\\
$\pi^u(\Sigma \in \mathcal{U}(\zeta)) \geq \underline{\pi^u(\zeta^{-1}\leq\min\limits_{j}(\sigma_{jj}-\zeta^{-1})\leq 2\max\limits_{j}\sigma_{jj}\leq\zeta)} \ * \ \underline{\pi^u(\max\limits_{k \neq j}|\sigma_{kj}|< (\zeta p)^{-1})}$\\
\\
\\
먼저 첫 번째 밑줄 확률을 계산해보면,\\
\\
$\pi^u(\zeta^{-1}\leq\min\limits_{j}(\sigma_{jj}-\zeta^{-1})\leq 2\max\limits_{j}\sigma_{jj}\leq\zeta)$ $\geq \pi^u(2\zeta^{-1} \leq \sigma_{jj} \leq \dfrac{\zeta}{2}, \forall j)$\\
$\geq \prod\limits_{j=1}^p \pi^u(2\zeta^{-1}\leq \sigma_{jj} \leq \dfrac{\zeta}{2})$ 에서,\\
$\sigma_{jj}$가 $\Gamma(1,\dfrac{\lambda}{2})$ 를 따른다는 가정에 의해, $f(\sigma_{jj})=\dfrac{\lambda}{2}exp(-\dfrac{\lambda}{2}\sigma_{jj})$ 의 pdf 를 갖고, 가로 길이가 $\left(\dfrac{2}{\zeta},\dfrac{\zeta}{2}\right)$이고 세로 길이가 $\left(0,f\left(\dfrac{2}{\zeta}\right)\right)$ 인 직사각형을 생각하면, \\
\\
이는 pdf 의 전체 넓이 보다는 작으므로, 이를 통해 $\prod\limits_{j=1}^p\pi^u(2\zeta^{-1}\leq \sigma_{jj} \leq \dfrac{\zeta}{2})$\\
$=\left\{\left(\dfrac{\zeta}{2}-\dfrac{2}{\zeta}\right)\dfrac{\lambda}{2}exp(-\dfrac{\lambda \zeta}{4})\right\}^p \geq \left\{ \dfrac{\lambda \zeta}{8} exp(-\dfrac{\lambda \zeta}{4})\right\}^p$ 임을 알 수 있다.\\
\\
\\
이제 두 번째 밑줄 확률인 $\pi^u(\max\limits_{k \neq j}|\sigma_{kj}|< (\zeta p)^{-1})$를 계산할 건데, 먼저 이를 위한 lemma 하나를 소개하자.\\
\\
\ \\
\noindent
[lemma 1 in [12]]\\
\\
The univariate horseshoe density $p(\theta)$ satisfies the following:
\begin{align*}
    &(a) \lim\limits_{\theta \rightarrow 0}p(\theta) = \infty\\
    &(b) \ For\ \theta \neq 0, \dfrac{K}{2}log\left(1+\dfrac{4}{\theta^2}\right)<p(\theta)<Klog\left(1+\dfrac{2}{\theta^2}\right), \ where\ K=\dfrac{1}{\sqrt{2\pi^3}}
\end{align*} 

\noindent
따라서, 위의 lemma\ (b) 를 통해, $\pi^u(\max\limits_{k \neq j}|\sigma_{kj}|< (\zeta p)^{-1})$의 계산을 위한 부등식인 $\pi^u(\sigma_{kj})\leq \dfrac{1}{\tau \sqrt{2\pi^3}}ln\left(1+\dfrac{2\tau^2}{\sigma_{kj}^2}\right)$ 를 알 수 있다.\\
\\
한편, $|\sigma_{kj}|$ 는 이대일 변환이므로, 2가 곱해져서,\\
\\
$\pi^u(|\sigma_{kj}|\geq(\zeta p)^{-1})\leq \dfrac{1}{\zeta}\sqrt{\dfrac{2}{\pi^3}} \displaystyle \int_{(\zeta p)^{-1}}^{\infty}ln\left(1+\dfrac{2\tau^2}{x^2}\right) dx$ 가 되고,\\
$\leq \sqrt{\dfrac{2}{\pi^3}} \displaystyle \int_{(\zeta p)^{-1}}^{\infty}\dfrac{2\tau^2}{x^2} dx$ $( \because ln(1+x) \leq x, \ when\ x\geq 0)$\\
\\
$= \sqrt{\dfrac{2}{\pi^3}} \displaystyle \int_{(\zeta p)^{-1}}^{\infty}\dfrac{2\tau}{x^2} dx$
$=\dfrac{2\sqrt{2}}{\sqrt{\pi^3}}\tau\zeta p $ 임을 알 수 있다.\\
\\
\\
이를 통해 이제 $\pi^u(\max\limits_{k \neq j}|\sigma_{kj}|< (\zeta p)^{-1})$ 를 계산해보면,\\
\\
$\pi^u(\max\limits_{k \neq j}|\sigma_{kj}|< (\zeta p)^{-1}) = \prod\limits_{k\neq j} \left\{    1-\pi^u(|\sigma_{kj}|\geq (\zeta p)^{-1})   \right\} = \left(1-\dfrac{2\sqrt{2}}{\sqrt{\pi^3}}\tau\zeta p\right)^{p(p-1)}$ \\
$\geq \left(1-\dfrac{2\sqrt{2}}{\sqrt{\pi^3}}\tau\zeta p\right)^{p^2} \ \cdots \ (1) $ $(\because$ 괄호안의 값은확률로 1보다 작으므로)\\
\\
\\
$\geq exp\left(-\dfrac{4\sqrt{2}}{\sqrt{\pi^3}}\tau \zeta p^3\right) \ (\because log(1-x) \geq -2x,\ when\ x\leq \dfrac{1}{2})$ 이다.\\
한편, 주어진 조건에서 $\tau=O\left(\dfrac{1}{p^2}\sqrt{\dfrac{s_0lnp}{n}}\right), s_0lnp=O(n)$ 이라 했으므로,\\
$\dfrac{\tau}{\dfrac{1}{p^2}\sqrt{\dfrac{s_olnp}{n}}}\leq c_1, \dfrac{s_olnp}{n}\leq c_2,\ c_1,c_2>0$ 임을 알 수 있다. 이들을 조합하면\\
\\
$\Rightarrow \ \tau p^2 \leq \sqrt{c_2}c_1 \defeq c_3 \ \Rightarrow \tau p^3 \leq c_3p\ \Rightarrow \ -\tau p^3 \geq -c_3p $\ 이다. 이를 (1) 식에서 활용하면, $exp\left(-\dfrac{4\sqrt{2}}{\sqrt{\pi^3}}\tau \zeta p^3\right) \geq exp\left(-c_3 \dfrac{4\sqrt{2}}{\sqrt{\pi^3}} \zeta p\right)=exp(-Cp)$ 이다. 따라서, 두번째 밑줄 확률의 부등식을 다음과 같이 구할 수 있다.\\
\\
$\pi^u(\max\limits_{k \neq j}|\sigma_{kj}|< (\zeta p)^{-1}) \geq exp(-Cp) $ \\
\\
이제 첫 번째 밑줄 식과 두 번째 밑줄 식의 결과를 종합하면,\\
\\
$\pi^u(\Sigma \in \mathcal{U}(\zeta)) \geq \pi^u(\zeta^{-1}\leq\min\limits_{j}(\sigma_{jj}-\zeta^{-1})\leq 2\max\limits_{j}\sigma_{jj}\leq\zeta) \ * \ \pi^u(\max\limits_{k \neq j}|\sigma_{kj}|< (\zeta p)^{-1})$\\
\\
$\geq \left\{ \dfrac{\lambda \zeta}{8} exp(-\dfrac{\lambda \zeta}{4})\right\}^p exp(-Cp)=\left\{ \dfrac{\lambda \zeta}{8} exp(-\dfrac{\lambda \zeta}{4}-C)\right\}^p$ 임을 알 수 있다.\\








\end{proof}


\section{오정훈 - Theorem 4 and Lemma 2}

본 절에서는 \cite{lee2022beta} 의 사후 수렴속도에 관한 증명인 \textbf{Lemma 1} 을 보이기 위하여 필요한 두번째 조건 \textbf{Theorem 4} 를 증명하고, \textbf{Lemma 4} 의 증명에 필요한 \textbf{Lemma 3} 를 증명한다.

\setcounter{theorem}{4}

\begin{theorem}[The Upper Bound of the Packing Number] If $\zeta^4 \leq p, \; p \asymp n^{\beta}$ for some $0 < \beta < 1, \; s_n = c_1 n \epsilon_n^{2} / \ln{p}, \; L_n = c_2 n \epsilon_n^2$ and $\delta_n = \epsilon_n / \zeta^3$ for some constants $c_1 > 1$ and $c_2 > 0,$ we have
    \begin{equation*}
        \ln{ D(\epsilon_n, \mathcal{P}_n, d) } \leq (12 + 1/\beta) c_1 n \epsilon_n^2.
    \end{equation*}
\end{theorem}

\begin{proof}
    정의 $\mathcal{P}_n = \{ f_{\Sigma} : |s(\Sigma, \delta_n)| \leq s_n, \; \zeta^{-1} \leq \lambda_{min}(\Sigma) \leq \lambda_{max}(\Sigma) \leq \zeta, \; || \Sigma ||_{max} \leq L_n \}$로부터, 가산반가법성에 의해 다음이 성립한다.
    \begin{equation}
      \pi(\mathcal{P}_n^c) \leq \pi(|s(\Sigma, \delta_n)| > s_n) + \pi(||\Sigma||_{max} > L_n)  
    \end{equation}
    부등식 (1)의 첫번째 항의 상계를 구해보자. 앞선 \textbf{Lemma 2}의 증명의 결과를 따라가는데, 모형 $\sigma_{kj} | \lambda \sim \mathcal{N}(0, \lambda^2), \; \lambda \sim C^{+}(0, \tau)$ 에 \cite{carvalho2010horseshoe} 의 \textbf{Theorem 1} 의 부등식을 적용하면
    \begin{equation}
      \pi^u(\sigma_{kj}) \leq \frac{1}{\tau \sqrt{2 \pi^3}} \ln{ \left( 1 + \frac{2 \tau^2}{\sigma_{kj}^2} \right)}, \quad 1 \leq k \neq j \leq p
    \end{equation}
    가 성립하므로,
    \begin{equation}
      \pi^u(|\sigma_{kj}| > \delta_n) \leq \frac{2}{\tau \sqrt{2 \pi^3}} \int_{\delta_n}^{\infty} \ln{ \left( 1 + \frac{2 \tau^2}{x^2} \right)} dx \leq \frac{\sqrt{2}}{\tau \sqrt{\pi^3}} \int_{\delta_n}^{\infty} \frac{2\tau^2}{x^2} dx = \frac{2 \sqrt{2}}{\sqrt{\pi^3}} \tau \delta_n^{-1}.
    \end{equation}
    를 얻는다. 두번째 부등식에서는 $\ln{(1 + x)} \leq x$ 를 이용하였다. 이제 (3)으로부터, 다음의 사실을 관찰할 수 있다. \\
    적당한 상수 $C > 0$에 대해,
    \begin{equation}
      \nu_n \equiv \pi^u (|\sigma_{kj}| > \delta_n) \leq \frac{2 \sqrt{2}}{\sqrt{\pi^3}} \tau \delta_n^{-1} = \frac{2\sqrt{2}}{\sqrt{\pi^3}} \tau \frac{\sqrt{n} \zeta^3}{\sqrt{(p+s_0)\ln{p}}} \leq \frac{C}{p^2} \sqrt{\frac{s_0 \ln{p}}{n}} \frac{\sqrt{n} \zeta^3}{\sqrt{(p+s_0)\ln{p}}} \lesssim \frac{1}{p^2}
    \end{equation}
    이 충분히 큰 모든 $n$에 대해 성립한다.  \\
    두번째 등식은 가정 $\delta_n = \epsilon_n/\zeta^3$을 대입하였고, 세번째 부등식은 $\tau = O(\frac{1}{p^2} \sqrt{\frac{s_0 \ln{p}}{n}})$ 을 사용하였다.
    이제까지의 결과를 정리하면, 우리는
    \begin{equation}
      \nu_n = \pi^u ( |\sigma_{kj}| > \delta_n ) \lesssim \frac{1}{p^2}
    \end{equation}
    의 사실을 얻었다. 이제 위에서 정의된 것들을 살펴보자. $\nu_n$이 0이 아닌 비대각 원소 $\sigma_{kj}$들의 비율이고, $|s(\Sigma, \delta_n)|$이 0이 아닌 $\sigma_{kj}$의 개수라는 해석이 자연스럽다. 따라서 $|s(\Sigma, \delta_n)| \sim Bin( {p \choose 2}, \nu_n)$을 따른다. 여기에 \cite{song2017nearly} 의 \textbf{Lemma A.3} 의 결과를 이용하면,
    \begin{eqnarray}
      \pi^u(|s(\Sigma, \delta_n)| > s_n) = \mathbb{P} \left( Bin({p \choose 2}, \nu_n) > s_n \right) &\leq& 1 - \Phi \left( sign(s_n - {p \choose 2} \nu_n) \sqrt{2 {p \choose 2} H( \nu_n, s_n / {p \choose 2})} \right) \nonumber \\
      & = & 1 - \Phi \left( \sqrt{2 {p \choose 2} H( \nu_n, s_n / {p \choose 2})} \right)
    \end{eqnarray}
    를 얻는다. (6)의 마지막 등식에서는 최대값이 평균보다 크다는 사실, $s_n \geq {p \choose 2} \nu_n$ 을 사용하였다. $\Phi$는 표준정규분포의 cdf이고, $H$는 다음과 같이 정의된 함수이다.
    \[
    H(\nu, \frac{k}{n}) := \frac{k}{n} \ln{ \left( \frac{k}{n \nu} \right) } + \left( 1 - \frac{k}{n} \right) \ln{\left( \frac{1 - \frac{k}{n}}{1 - \nu} \right)}
    \]
    이제 표준정규분포의 cdf에 관한 부등식 $1 - \Phi (t) \leq \frac{1}{\sqrt{2\pi}} \frac{1}{t} e^{-\frac{t^2}{2}}$ 을 사용하면,
    \begin{equation}
      1 - \Phi \left( \sqrt{2 {p \choose 2} H ( \nu_n, s_n / {p \choose 2})} \right) \leq \frac{e^{- {p \choose 2}H(\nu_n, s_n/{p \choose 2})}}{\sqrt{2\pi}\sqrt{2{p \choose 2}H(\nu_n, s_n/{p \choose 2})}}
    \end{equation}
    의 상계를 얻는다. (7)의 점근적 성질을 확인해보기 위해, 항 ${p \choose 2} H(\nu_n, s_n/{p \choose 2})$을 전개해보자.
    \begin{equation}
      {p \choose 2} H (\nu_n, s_n/{p \choose 2}) = s_n \ln{\left( \frac{s_n}{{p \choose 2} \nu_n} \right)} + \left( {p \choose 2} - s_n \right) \ln{\left( \frac{{p \choose 2} - s_n}{{p \choose 2} - {p \choose 2}\nu_n} \right)}
    \end{equation}
    (8)의 첫번째 항을 먼저 보면, 적당한 상수 $C > 0$이 존재하여
    \begin{eqnarray}
      s_n \ln{\left( \frac{s_n}{{p \choose 2} \nu_n} \right)} = s_n \ln{\left( \frac{c_1(p + s_0)}{{p \choose 2}\nu_n} \right)} \geq s_n \ln{\left( c_1 C (p + s_0) \right)} \geq s_n \ln{\left( \sqrt{p + s_0} \right)} &\geq& \frac{s_n}{2} \ln{p}  \\
      &=& \frac{c_1 n \epsilon_n^2}{2}
    \end{eqnarray}
    이 충분히 큰 모든 $n$에 대하여 성립함을 알 수 있다. (9)의 첫번째 등식은 $s_n = c_1 n \epsilon_n^2 / \ln{p} = c_1(p + s_0)$를 이용하였고, 두번째 부등식에서는 (5)의 결과 $\nu_n \lesssim \frac{1}{p^2}$를 사용하였다. 다음으로 (8)의 두번째 항을 살펴보면
    \begin{eqnarray}
      \left( {p \choose 2} - s_n \right) \ln{\left( \frac{{p \choose 2} - s_n}{{p \choose 2} - {p \choose 2}\nu_n} \right)} &=& \left( {p \choose 2} - s_n \right) \ln{\left( 1 - \frac{s_n - {p \choose 2} \nu_n}{{p \choose 2}(1 - \nu_n)} \right)} \\
      &\geq& -2 \left( {p \choose 2} - s_n \right) \frac{s_n - {p \choose 2} \nu_n}{{p \choose 2}(1 - \nu_n)} \\
      &=& -2 \left( 1 - s_n / {p \choose 2} \right) \frac{s_n - {p \choose 2} \nu_n }{(1 - \nu_n)}  \\
      &\geq& -2 \left( 1 - s_n / {p \choose 2} \right) \frac{s_n }{(1 - \nu_n)} \\
      &=& -s_n \left( 1 - s_n / {p \choose 2} \right) \frac{2}{(1 - \nu_n)} \\
      &\gtrsim& -s_n \left( 1 - \frac{c_1(p + s_0)}{p^2} \right)  \\
      &\gtrsim& -s_n = \frac{- c_1 n \epsilon_n^2}{\ln{p}}
    \end{eqnarray}
    이 충분히 큰 모든 $n$에 대해 성립한다. 부등식 (12)는 $\ln{(1-x)} \geq -2x, \; \forall x \in (0, 1/2)$과 $\nu_n \lesssim 1/p^2$로부터 $\frac{s_n - {p \choose 2}\nu_n}{{p \choose 2}(1 - \nu_n)} =  O(\frac{p + s_0}{p^2}) \searrow 0, \; n \rightarrow \infty$임을 이용하였다. 부등식 (16)은 $s_n/{p \choose 2} = c_1(p + s_0) / {p \choose 2} = O(\frac{c_1(p + s_0)}{p^2})$인 사실과 $\nu_n \lesssim 1/p^2$로부터 충분히 큰 모든 n에 대해 $\frac{2}{1-\nu_n}$이 상수로 무시가능한 것으로 인해 성립하고, (17)번 부등식은 $1 - \frac{c_1(p + s_0)}{p^2} \rightarrow 1$ 때문에 성립한다. \\
    (10)과 (17)의 결과를 가지고 (8)로 되돌아가면 다음을 관찰할수 있다.
    \begin{align}
      {p \choose 2} H (\nu_n, s_n/{p \choose 2}) \gtrsim \frac{c_1 n \epsilon_n^2}{2} - \frac{c_1 n \epsilon_n^2}{\ln{p}} = O(\frac{c_1 n \epsilon_n^2}{2})
    \end{align}
    즉 ${p \choose 2} H (\nu_n, s_n/{p \choose 2})$ 은 무한으로 발산하므로, 충분히 큰 모든 $n$에 대하여 (7) 우변의 분모를 생략할 수 있다. 그리고 적당히 큰 $n$에 대해,
    \begin{align}
      {p \choose 2} H (\nu_n, s_n/{p \choose 2}) \gtrsim \left( \frac{1}{2} - \frac{1}{\ln{p}} \right) c_1 n \epsilon_n^2 \sim \frac{1}{3} c_1 n \epsilon_n^2
    \end{align}
    역시 성립하므로 (18)과 (19)의 결과들을 (6), (7)과 결합하면, 충분히 큰 모든 $n$에 대하여
    \begin{align}
      \pi^u(|s(\Sigma, \delta_n)| > s_n) \leq \exp \left( - \frac{c_1 n \epsilon_n^2}{3} \right)
    \end{align}
    이 성립함을 알 수 있다. 여태까지의 모든 결과를 종합하면 (1)에서 제시된 첫번째 항의 상계를 다음과 같이 구할 수 있다.
    \begin{eqnarray}
      \pi(|s(\Sigma, \delta_n)| > s_n) &\leq& \frac{\pi^u(|s(\Sigma, \delta_n)| > s_n)}{\pi^u(\Sigma \in \mathcal{U}(\zeta))} \leq \frac{\exp \left( -c_1 n \epsilon_n^2 /3 \right)}{\pi^u(\Sigma \in \mathcal{U}(\zeta))}  \\
      &\leq& \left\{ \frac{8}{\lambda \zeta} \exp( \frac{\lambda \zeta}{4} + C) \right\}^p \exp(-c_1 n \epsilon_n^2 /3) \\
      &=& \exp \left( p \ln{8} - p \ln{(\lambda \zeta)} + p \frac{\lambda \zeta}{4} + Cp - \frac{c_1 n \epsilon_n^2}{3} \right) \\
      &\lesssim& \exp \left( p \ln{p} - \frac{c_1 n \epsilon_n^2}{3} \right)  \\
      &\sim& \exp \left( - \frac{(c_1 - 1) n \epsilon_n^2}{3} \right)
    \end{eqnarray}
    이 충분히 큰 모든 $n$에 대해 성립한다. (21)의 두번째 부등식에서는 (20)의 결과를, 부등식 (22)에서는 \cite{lee2022beta}의 \textbf{Lemma 2}를, (24)에서는 $\lambda \zeta \leq \ln{p}$의 가정을 사용하였다. \\
    마지막으로 (1)의 두번째 항인 $\pi ( ||\Sigma||_{max} > L_n)$의 상계를 구하여보자. 먼저 $||\Sigma||_{max} \leq \lambda_{max}(\Sigma)$ 가 성립하는데, 이는 다음으로부터 알 수 있다. 실수인 행렬 $A$가 양의 준정부호 혹은 정부호 행렬일때,
    \begin{eqnarray}
      \lambda_{max}(A) = \max_{||\vectorbold{u}||=1} \vectorbold{u}^\top A \vectorbold{u} \geq \vectorbold{e}_i^\top A \vectorbold{e}_i, \quad \forall i=1,\cdots,p
    \end{eqnarray}
    가 성립. $\{ \vectorbold{e}_1, \cdots, \vectorbold{e}_p \}$는 $\mathbb{R}^p$의 표준기저이다. 그런데 $A$가 대칭이며 양의 준정부호이면 $A$의 최대원소는 반드시 대각원소중에 있으므로 $||\Sigma||_{max} \leq \lambda_{max}(\Sigma).$ 그리고 $\pi(\Sigma : \lambda_{max}(\Sigma) \leq \zeta) = 1$으로부터, 충분히 큰 모든 $n$에 대하여 $L_n > \zeta$이면,
    \begin{equation}
      0 = \pi(\lambda_{max}(\Sigma) > L_n) \geq \pi(||\Sigma||_{max} > L_n)
    \end{equation}
    이 성립한다. 그러므로 (25)와 (27)에 의해 (1)은
    \begin{eqnarray}
    \pi(\mathcal{P}_n^c) &\leq& \pi(|s(\Sigma,\delta_n)| > s_n) + \pi(||\Sigma||_{max} > L_n) \\
    &\leq& \exp \left( - \frac{(c_1 - 1) n \epsilon_n^2}{3} \right)
  \end{eqnarray}
  가 충분히 큰 모든 $n$에 대해 성립한다. 이로써 증명이 끝났다.
  \end{proof}


  % \setcounter{lemma}{2}
  \begin{lemma} If $\Sigma_0 \in \mathcal{U}(s_0, \zeta_0)$ and $\Sigma \in \mathcal{U}(\zeta)$, we have
    \begin{enumerate}[(i)]
      \item $K(f_{\Sigma_0}, f_\Sigma) \leq c_0 \zeta^4 \zeta_0^2 || \Sigma - \Sigma_0 ||_F^2$ \; \text{for some } $c_0 > 0$
      \item $V(f_{\Sigma_0}, f_\Sigma) \leq \frac{3}{2} \zeta^4 \zeta_0^2 || \Sigma - \Sigma_0 ||_F^2 \; $ for sufficiently small $|| \Sigma - \Sigma||_F^2 \leq \frac{1}{c_0^2 \zeta^4 \zeta_0^2}.$
    \end{enumerate}    
  \end{lemma}

  \begin{proof}
  \textbf{Lemma 3}의 증명은 \cite{banerjee2015bayesian}의 계산을 따라간다. 먼저 다음을 확인하자. $A = \Sigma_0^{\frac{1}{2}} \Sigma^{-1} \Sigma_0^{\frac{1}{2}}$라 놓고, $d_i$를 $A$의 고유값이라 하자. $d_i = \lambda_i(A), \; i=1,\cdots,p.$ 또, $D = diag(d_i, \; i=1,\cdots,p)$라 놓는다. 그러면
  \begin{align*}
    ||I - A||_F^2 = tr \left[ (I - A)^2 \right] = tr \left[ U(I - D)^2 U^\top \right] = tr \left[ (I - D)^2 \right] = \sum_{i=1}^{p} (1 - d_i)^2, \quad UU^\top = U^\top U = I_p
  \end{align*}
  이다. 첫번째 등식은 $|| B ||_F^2 = tr \left( B^\top B \right) = tr \left( B B^\top \right)$임을, 두번째 등식은 A의 고유치 분해를 사용하였다. 이로써
  \begin{eqnarray}
      \sum_{i=1}^n (1 - d_i)^2 &=&|| I - A ||_F^2  \\
      &=& || \Sigma_0^{\frac{1}{2}} ( \Sigma_0^{-1} - \Sigma^{-1} ) \Sigma_0^{\frac{1}{2}} ||_F^2 \\
      &\leq& || \Sigma_0^{\frac{1}{2}} ||^2 \; || \Sigma_0^{\frac{1}{2}} ||^2 \; || \Sigma_0^{-1} - \Sigma^{-1} ||_F^2  \\
      &=& \lambda_{max}(\Sigma_0) \cdot \lambda_{max}(\Sigma_0) \cdot || \Sigma_0^{-1} - \Sigma^{-1} ||_F^2 \\
      &\leq& \zeta_0^2 \cdot || \Sigma_0^{-1} - \Sigma^{-1} ||_F^2  \\
      &\leq& \zeta^2 \cdot || \Sigma_0^{-1} - \Sigma^{-1} ||_F^2
    \end{eqnarray}
    가 성립함을 알 수 있다. (31) 부등식은 아래의 \textbf{Lemma 5}의 결과를 이용하였고, 부등식 (35)는 \cite{lee2022beta} p.5의 가정 \textbf{A3.} $\zeta > \max{(3, \zeta_0)}$로부터 성립한다.
    위와 비슷한 방식으로, 
    \begin{align}
      || \Sigma_0^{-1} - \Sigma^{-1} ||_F \leq || \Sigma^{-1} || \; || \Sigma_0^{-1} || \; || \Sigma - \Sigma_0 ||_F \leq \zeta \zeta_0 || \Sigma - \Sigma ||_F
    \end{align}
    가 성립한다. 먼저 (i)의 쿨벡 라이블러 발산을 계산한다. $\mathbb{E}_0$를 밀도함수 $f_{\Sigma_0}$에 대한 기대값이라 하자.
    \begin{eqnarray}
      K(f_{\Sigma_0}, f_\Sigma) &=& \mathbb{E}_0 \left[ -\frac{1}{2} \ln{\frac{|\Sigma_0|}{|\Sigma|}} - \frac{1}{2} x^\top (\Sigma_0^{-1} - \Sigma^{-1}) x \right], \quad x \sim \mathcal{N}(0, \Sigma_0) \\
      &=& - \frac{1}{2} \ln{\frac{|\Sigma_0|}{|\Sigma|}} - \frac{1}{2} tr \left[ (\Sigma_0^{-1} - \Sigma^{-1}) \Sigma_0 \right] \\
      &=& - \frac{1}{2} \ln{(|\Sigma_0^{\frac{1}{2}}| |\Sigma|^{-1} |\Sigma_0^{\frac{1}{2}}|)} - \frac{1}{2} tr(I - \Sigma^{-1}\Sigma_0)  \\
      &=& - \frac{1}{2} \ln{|A|} - \frac{1}{2} tr \left[ \Sigma_0^{\frac{1}{2}} (I - A) \Sigma_0^{-\frac{1}{2}} \right] \\
      &=& - \frac{1}{2} \ln{|A|} - \frac{1}{2} tr(I -A) \\
      &=& - \frac{1}{2} \sum_{i=1}^{p} \ln{d_i} - \frac{1}{2} \sum_{i=1}^{p} (1 - d_i)
    \end{eqnarray}
    등식 (38)에서는 다변량 정규분포의 이차형식의 기대값에 관한 공식을 사용하였다.
    다음으로 (ii)의 쿨벡 라이블러 분산을 계산해보면
    \begin{eqnarray}
      V(f_{\Sigma_0}, f_\Sigma) &=& \mathbb{E}_0 \left[ \frac{1}{4} \left\{ \ln{\frac{|\Sigma_0|}{|\Sigma|}} + x^\top (\Sigma_0^{-1} - \Sigma^{-1}) x \right\}^2 \right], \quad z \stackrel{let}{=} x^\top (\Sigma_0^{-1} - \Sigma^{-1}) x  \\
      &=& \frac{1}{4} \; \mathbb{E}_0 \left[ \left\{ \ln{\frac{|\Sigma_0|}{|\Sigma|}} + \mathbb{E}_0 (z) + z - \mathbb{E}_0 (z) \right\}^2 \right] \\
      &=& \frac{1}{4} \; \bigg[ 4 \big\{ K (f_{\Sigma_0}, f_\Sigma) \big\}^2 + Var(z) \bigg]  \\
      &=& K^2 (f_{\Sigma_0}, f_\Sigma) + \frac{1}{2} tr \bigg[ (\Sigma_0^{-1} - \Sigma^{-1}) \Sigma_0 (\Sigma_0^{-1} - \Sigma^{-1}) \Sigma_0 \bigg] \\
      &=& K^2 (f_{\Sigma_0}, f_\Sigma) + \frac{1}{2} tr \left[ \Sigma_0^{\frac{1}{2}} (\Sigma_0^{-1} - \Sigma^{-1}) \Sigma_0^{\frac{1}{2}} \Sigma_0^{\frac{1}{2}} (\Sigma_0^{-1} - \Sigma^{-1}) \Sigma_0^{\frac{1}{2}} \right]  \\
      &=& K^2 (f_{\Sigma_0}, f_\Sigma) + \frac{1}{2} tr \big[ (I - A)^2 \big] \\
      &=& K^2 (f_{\Sigma_0}, f_\Sigma) + \frac{1}{2} \sum_{i=1}^{p} (1 - d_i)^2
    \end{eqnarray}
    등식 (45)에서는 등식 (37)과 다변량 정규분포의 이차형식의 분산에 관한 공식을 사용하였다. 이제 부등식 (i)을 보인다.
    \begin{eqnarray}
      K(f_{\Sigma_0}, f_\Sigma) &=& - \frac{1}{2} \sum_{i=1}^{p} \ln{d_i} - \frac{1}{2} \sum_{i=1}^{p} (1 - d_i)  \\
      &\leq& c_0 \sum_{i=1}^{p} (1 - d_i)^2 \quad \text{for some } c_0 > 0  \\
      &\leq& c_0 \zeta^2 || \Sigma_0^{-1} - \Sigma^{-1} ||_F^2  \\
      &\leq& c_0 \zeta^4 \zeta_0^2 || \Sigma_0 - \Sigma ||_F^2
    \end{eqnarray}
    부등식 (51)은 아래의 \textbf{참고 1}을 보라. 부등식 (52)는 (35), 부등식 (53)은 (36)의 결과를 이용하였다.
    \begin{remark}
    먼저, 다음과 같은 부등식이 성립함을 알 수 있는데,
    \[
      -\frac{1}{2} \ln{x} -\frac{1}{2}(1 - x) \leq (1 - x)^2, \; \forall x \in [\eta_0, 1] \; \text{ for small } \eta_0 > 0.
    \]
    여기서 좌변과 우변의 두 함수는 각각 $\eta_0$과 $1$에서 만남을 확인가능하다. 또, A의 고유값 $d_i$에 대하여
    \[
    d_i = \lambda_i(A) = \lambda_i(\Sigma^{-1} \Sigma_0) \geq \lambda_{min}(\Sigma_0) \lambda_{min}(\Sigma^{-1}) \geq \zeta_0^{-1} \zeta^{-1}
    \]
    이 성립함을 알 수 있는데, 다음 등식을 만족하는 $c_0$를 잡아오는 것을 생각해보자. $x \neq 1$이면
    \[
    c_0 (1 - x)^2 = -\frac{1}{2} \ln{x} - \frac{1}{2} (1 - x) \iff c_0 = - \frac{\ln{x} + 1 - x}{2(1 - x)^2} > 0
    \]
    위에서 $d_i \geq \zeta_0^{-1} \zeta^{-1}$을 알고 있으므로
    \[
    c_0 = - \frac{\ln{\zeta_0^{-1}\zeta^{-1}} + 1 - \zeta_0^{-1}\zeta^{-1}}{2(1 - \zeta_0^{-1}\zeta^{-1})^2}  > 0
    \]
    과 같이 잡으면, (51)이 성립한다.
    \end{remark}
    마지막으로, 부등식 (ii)를 보인다. 위에서 보인 결과들을 종합하면
    \begin{eqnarray}
      V(f_{\Sigma_0}, f_\Sigma) &=& \frac{1}{2} \sum_{i=1}^{p} (1 - d_i)^2 + K^2 (f_{\Sigma_0}, f_\Sigma) \\
      &\leq& \frac{1}{2} \sum_{i=1}^{p} (1 - d_i)^2 + \zeta^4 \zeta_0^2 || \Sigma - \Sigma_0 ||_F^2 \\
      &\leq& \frac{3}{2} \zeta^4 \zeta_0^2 || \Sigma - \Sigma_0 ||_F^2
    \end{eqnarray}
    가 충분히 작은 $|| \Sigma - \Sigma_0 ||_F^2 \leq 1/(c_0^2 \zeta^4 \zeta_0^2)$에 대해 성립한다.  \\
    부등식 (55)는 위에서 보인 결과 (i)로부터 $|| \Sigma - \Sigma_0 ||_F^2 \leq 1/(c_0^2 \zeta^4 \zeta_0^2)$로 충분히 작으면, $K^2(f_{\Sigma_0}, f_{\Sigma}) \leq c_0^2 \zeta^8 \zeta_0^4 || \Sigma - \Sigma_0 ||_F^4 \leq \zeta^4 \zeta_0^2 || \Sigma - \Sigma_0 ||_F^2$인 것을 사용하였고, 부등식 (56)은 부등식 (35)와 (36)의 결과에 의해 성립한다. 이로써 \textbf{Lemma 3}가 증명되었다.
    
  \end{proof}

  % \setcounter{lemma}{4}

  \begin{lemma}[\cite{lee2022beta}]
  For any $p \times p$ matrices $A$ and $B$, we have
  \[
  ||AB||_F^2 \leq ||A|| \; ||B||_F, \quad ||AB||_F^2 \leq ||A||_F \; ||B||
  \]
  \end{lemma}
  \begin{proof}
    $\vectorbold{b}_j$가 행렬 $B$의 $j$번째 열이라고 하자. 그러면
    \begin{align*}
      ||AB||_F^2 = ||(A\vectorbold{b}_1, \cdots, A\vectorbold{b}_p)||_F^2 = \sum_{j=1}^{p} ||A\vectorbold{b}_j||_2^2 \leq ||A||^2 \sum_{j=1}^{p}||\vectorbold{b}_j||_2^2 = ||A||^2 \; ||B||_F^2
    \end{align*}
    마찬가지로 $\vectorbold{a}_i^\top$가 행렬 $A$ 의 $i$번째 행이라고 하면
    \begin{align*}
      ||AB||_F^2 = ||(\vectorbold{a}_1^\top B, \cdots, \vectorbold{a}_p^\top B)||_F^2 = \sum_{i=1}^{p} ||\vectorbold{a}_i^\top B||_2^2 \leq ||B||^2 \sum_{i=1}^{p}||\vectorbold{a}_i||_2^2 = ||B||^2 \; ||A||_F^2
    \end{align*}
  \end{proof}


\section{장태영 - Lemma 4 and Theorem 5}

% \documentclass[12pt]{article}
% \usepackage[utf8]{inputenc}
% \usepackage{latexsym, amssymb, amscd, amsxtra, amsmath, amsthm}
% \usepackage{graphics, graphicx, color}
% \usepackage{geometry}
% \geometry{
% 	a4paper,
% 	left=15mm,
% 	right=15mm,
% 	top=30mm,
% 	bottom=30mm
% }
% \usepackage{multibib}
% \usepackage{natbib}
% \usepackage{ifpdf}
% \usepackage[format=hang,indention=-1cm,small]{caption}
% % \usepackage[caption=false]{subfig}
% \usepackage[rightcaption]{sidecap}
% \usepackage{subcaption}
% \usepackage{multirow}
% \usepackage{kotex}
% \usepackage{graphicx}
% \graphicspath{ {./images/} }
% \usepackage{hyperref}
% \hypersetup{
%     colorlinks=true,
%     linkcolor=black,
%     citecolor=blue,
%     filecolor=magenta,      
%     urlcolor=blue,
%     pdfpagemode=FullScreen,
%     }
% \newcommand\numberthis{\addtocounter{equation}{1}\tag{\theequation}}
% % https://tex.stackexchange.com/questions/42726/align-but-show-one-equation-number-at-the-end}
% % align* environment but show one equation number at some lines


% % \title{Latex template for academic writing}
% % \author{Taeyoung Chang}
% % \date{\today}

% % MATH -----------------------------------------------------------
% \newcommand{\norm}[1]{\left\Vert#1\right\Vert}
% \newcommand{\abs}[1]{\left\vert#1\right\vert}
% \newcommand{\set}[1]{\left\{#1\right\}}
% \newcommand{\floor}[1]{\left\lfloor #1 \right\rfloor}
\newcommand{\Real}{\mathbb R}
% \newcommand{\Nat}{\mathbb N}
% \newcommand{\Int}{\mathbb Z}
% \newcommand{\Complex}{\mathbb C}
\newcommand{\eps}{\varepsilon}
% \newcommand{\To}{\longrightarrow}
% \newcommand{\E}{\text{E}}
% \newcommand{\V}{\text{Var}}
% \newcommand{\pr}{\text{Pr}}
% \newcommand{\I}{\text{I}}

% \newcommand{\tab}{\hspace*{1mm}}

% \def\av{\mathbf a}
% \def\bv{\mathbf b}
% \def\cv{\mathbf c}
% \def\dv{\mathbf d}
% \def\ev{\mathbf e}
% \def\fv{\mathbf f}
% \def\gv{\mathbf g}
% \def\hv{\mathbf h}
% \def\iv{\mathbf i}
% \def\jv{\mathbf j}
% \def\gv{\mathbf g}
% \def\kv{\mathbf k}
% \def\lv{\mathbf l}
% \def\mv{\mathbf m}
% \def\nv{\mathbf n}
% \def\ov{\mathbf o}
% \def\pv{\mathbf p}
% \def\qv{\mathbf q}
% \def\rv{\mathbf r}
% \def\sv{\mathbf s}
% \def\tv{\mathbf t}
% \def\uv{\mathbf u}
% \def\vv{\mathbf v}
% \def\wv{\mathbf w}
% \def\xv{\mathbf x}
% \def\yv{\mathbf y}
% \def\zv{\mathbf z}

% \def\Av{\mathbf A}
% \def\Bv{\mathbf B}
% \def\Cv{\mathbf C}
% \def\Dv{\mathbf D}
% \def\Ev{\mathbf E}
% \def\Fv{\mathbf F}
% \def\Gv{\mathbf G}
% \def\Hv{\mathbf H}
% \def\Iv{\mathbf I}
% \def\Jv{\mathbf J}
% \def\Kv{\mathbf K}
% \def\Lv{\mathbf L}
% \def\Mv{\mathbf M}
% \def\Nv{\mathbf N}
% \def\Ov{\mathbf O}
% \def\Pv{\mathbf P}
% \def\Qv{\mathbf Q}
% \def\Rv{\mathbf R}
% \def\Sv{\mathbf S}
% \def\Tv{\mathbf T}
% \def\Uv{\mathbf U}
% \def\Vv{\mathbf V}
% \def\Wv{\mathbf W}
% \def\Xv{\mathbf X}
% \def\Yv{\mathbf Y}
% \def\Zv{\mathbf Z}

% \newcommand{\alphav}{\mbox{\boldmath{$\alpha$}}}
% \newcommand{\betav}{\mbox{\boldmath{$\beta$}}}
% \newcommand{\gammav}{\mbox{\boldmath{$\gamma$}}}
% \newcommand{\deltav}{\mbox{\boldmath{$\delta$}}}
% \newcommand{\epsilonv}{\mbox{\boldmath{$\epsilon$}}}
% \newcommand{\zetav}{\mbox{\boldmath$\zeta$}}
% \newcommand{\etav}{\mbox{\boldmath{$\eta$}}}
% \newcommand{\thetav}{\mbox{\boldmath{$\theta$}}}
% \newcommand{\iotav}{\mbox{\boldmath{$\iota$}}}
% \newcommand{\kappav}{\mbox{\boldmath{$\kappa$}}}
% \newcommand{\lambdav}{\mbox{\boldmath{$\lambda$}}}
% \newcommand{\muv}{\mbox{\boldmath{$\mu$}}}
% \newcommand{\nuv}{\mbox{\boldmath{$\nu$}}}
% \newcommand{\xiv}{\mbox{\boldmath{$\xi$}}}
% \newcommand{\omicronv}{\mbox{\boldmath{$\omicron$}}}
% \newcommand{\piv}{\mbox{\boldmath{$\pi$}}}
% \newcommand{\rhov}{\mbox{\boldmath{$\rho$}}}
% \newcommand{\sigmav}{\mbox{\boldmath{$\sigma$}}}
% \newcommand{\tauv}{\mbox{\boldmath{$\tau$}}}
% \newcommand{\upsilonv}{\mbox{\boldmath{$\upsilon$}}}
% \newcommand{\phiv}{\mbox{\boldmath{$\phi$}}}
% \newcommand{\varphiv}{\mbox{\boldmath{$\varphi$}}}
% \newcommand{\chiv}{\mbox{\boldmath{$\chi$}}}
% \newcommand{\psiv}{\mbox{\boldmath{$\psi$}}}
% \newcommand{\omegav}{\mbox{\boldmath{$\omega$}}}
% \newcommand{\Sigmav}{\mbox{\boldmath{$\Sigma$}}}
% \newcommand{\Lambdav}{\mbox{\boldmath{$\Lambda$}}}
% \newcommand{\Deltav}{\mbox{\boldmath{$\Delta$}}}

% \newcommand{\Ac}{\mathcal{A}}
% \newcommand{\Bc}{\mathcal{B}}
\newcommand{\Cc}{\mathcal{C}}
% \newcommand{\Dc}{\mathcal{D}}
% \newcommand{\Ec}{\mathcal{E}}
% \newcommand{\Fc}{\mathcal{F}}
% \newcommand{\Gc}{\mathcal{G}}
% \newcommand{\Hc}{\mathcal{H}}
% \newcommand{\Ic}{\mathcal{I}}
% \newcommand{\Jc}{\mathcal{J}}
% \newcommand{\Kc}{\mathcal{K}}
% \newcommand{\Lc}{\mathcal{L}}
% \newcommand{\Mc}{\mathcal{M}}
% \newcommand{\Nc}{\mathcal{N}}
% \newcommand{\Oc}{\mathcal{O}}
% \newcommand{\Pc}{\mathcal{P}}
% \newcommand{\Qc}{\mathcal{Q}}
% \newcommand{\Rc}{\mathcal{R}}
% \newcommand{\Sc}{\mathcal{S}}
% \newcommand{\Tc}{\mathcal{T}}
\newcommand{\Uc}{\mathcal{U}}
% \newcommand{\Vc}{\mathcal{V}}
% \newcommand{\Wc}{\mathcal{W}}
% \newcommand{\Xc}{\mathcal{X}}
% \newcommand{\Yc}{\mathcal{Y}}
% \newcommand{\Zc}{\mathcal{Z}}

% \newtheorem{theorem}{Theorem}
% \newtheorem{corollary}{Corollary}
% \newtheorem{lemma}[theorem]{Lemma}

% % \newtheorem{theorem}{Theorem}
% \newtheorem*{theorem*}{Theorem}
% % \newtheorem{corollary}{Corollary}
% \newtheorem*{corollary*}{Corollary}
% % \newtheorem{lemma}{Lemma}
% \newtheorem*{lemma*}{Lemma}
% \newtheorem{remark}{Remark}
% \newtheorem*{remark*}{Remark}


% \begin{document}

% \begin{titlepage}
% 	\begin{center}
% 		\vspace*{5cm}
% 		\textbf{\Large Proof for the Theorems in the Article about Beta-Mixture Shrinkage Prior and Sparse Covariance}
% 		\\
% 		\vspace{1.5cm}
% 		\textbf{Taeyoung Chang}
% 		\vfill
%         The Beta-Mixture Shrinkage Prior for Sparse Covariances with Posterior Minimax Rates (2021)

% 		\vspace*{3cm}
% 		\thispagestyle{empty}
% 	\end{center}
% \end{titlepage}

\begin{lemma*}[Lemma 3 in the paper]
    If $\Sigma_0 \in \Uc(s_0, \zeta_0)$ and $\Sigma\in \Uc(\zeta)$ then we have 
    \begin{enumerate}
        \item $K(f_{\Sigma_0}, f_\Sigma)\leq \zeta^4\zeta_0^2 \norm{\Sigma-\Sigma_0}_F^2 $
        \item $V(f_{\Sigma_0}, f_\Sigma)\leq \frac 32 \zeta^4\zeta_0^2 \norm{\Sigma-\Sigma_0}_F^2 $
    \end{enumerate}
\end{lemma*}

\begin{lemma*}[Lemma 4 in the paper]
    If $a=b=1/2$ , $x>1$, and $\tau/x >0$ is sufficiently small then \[\pi_{ij}^u(x)\geq \sqrt{\frac{1}{2\pi}} \frac{\tau}{x^2}\] where $\pi_{ij}^u(\sigma_{ij})$ is the unconstrained marginal prior density of $\sigma_{ij}$
\end{lemma*}

\begin{theorem*}[Theorem 5 in the paper ; The lower bound for $\pi(B_{\eps_n})$]
    Here are the conditions we need for this theorem.
    \begin{enumerate}
        \item $\Sigma_0 \in \Uc(s_0, \zeta_0)$ with $\zeta_0 <\zeta$
        \item $p \asymp n^\beta$ for some $0<\beta<1$
        \item $\zeta^4\leq p$
        \item $\zeta^2\zeta_0^2 \leq s_0\log p$
        \item $n \geq \max \{1/\zeta_0^4\, , \,s_0/ (1-\zeta_0/\zeta)^2 \} \log p/\zeta^4$
        \item $p^{-1}<\lambda < \log p/\zeta_0$
        \item $a=b=1/2$
        \item $(p^2\sqrt{n})^{-1}\lesssim \tau \lesssim (p^2\sqrt{n})^{-1}\sqrt{s_0\log p}$ 
        \item (Additional, From Thm 1 at page 5 of the paper) \; $(p+s_0)\log p = o(n)$ i.e. $\eps_n^2 \rightarrow 0$
        \item (Additional, From page 4 of the paper) \; $p = O(s_0)$
    \end{enumerate}
    If the conditions above hold, then we have \[\pi(B_{\eps_n})\geq \exp\Big\{-(5+\frac 1\beta )n\eps_n^2 \Big\}\]
\end{theorem*}

\noindent \textbf{Proof of Lemma 4}
\begin{proof}
    Because we have $a=b=1/2$, \[\sigma_{ij}|\rho_{ij} \sim N(0, \frac{\rho_{ij}}{1-\rho_{ij}}\tau^2) \;,\; \rho_{ij} \sim \text{Beta}(a,b) \] is equivalent to \[\sigma_{ij}| \lambda_{ij} \sim N(0, \lambda_{ij}^2\tau^2)\;,\; \lambda_{ij} \sim \text{C}^+(0, 1)\] where $\text{C}^+(0, s)$ denotes the standard half-Cauchy distribution on positive real with a scale parameter $s$. 
    
    \begin{align*}
        p(\sigma, \rho) &= p(\sigma | \rho) p(\rho) = \frac{1}{\sqrt{2\pi \frac{\rho}{1-\rho}\tau^2}}\exp\Big(-\frac{1}{2\frac{\rho}{1-\rho}\tau^2}\sigma^2 \Big) \frac{1}{\pi}\rho^{-1/2}(1-\rho)^{-1/2} \quad \because \; \Gamma(1/2) = \sqrt{\pi}
    \end{align*}
    \begin{align*}
        \lambda &= \sqrt{\frac{\rho}{1-\rho}} \quad (\lambda>0)\quad \text{and} \quad \rho = \frac{\lambda^2}{\lambda^2+1} \\
        \text{Jacobian} &= \abs{\frac{d\rho}{d\lambda}} = \frac{2\lambda}{(\lambda^2+1)^2}
    \end{align*}
    \begin{align*}
        p(\sigma, \lambda) &= \frac{1}{\sqrt{2\pi \lambda^2 \tau^2}}\exp\Big(-\frac{1}{2\lambda^2 \tau^2}\sigma^2 \Big) \frac{1}{\pi} \sqrt{\frac{\lambda^2}{\lambda^2+1} \frac{1}{\lambda^2+1}}^{-1} \frac{2\lambda}{(\lambda^2+1)^2} \\
        &= \frac{1}{\sqrt{2\pi \lambda^2 \tau^2}}\exp\Big(-\frac{1}{2\lambda^2 \tau^2}\sigma^2 \Big) \frac{2}{\pi} \frac{\lambda^2+1}{\lambda} \frac{\lambda}{(\lambda^2+1)^2} \\
        &= \frac{1}{\sqrt{2\pi \lambda^2 \tau^2}}\exp\Big(-\frac{1}{2\lambda^2 \tau^2}\sigma^2 \Big) \frac{2}{\pi} \frac{1}{(\lambda^2+1)} \\
        &= p(\sigma|\lambda)p(\lambda)
    \end{align*}
    Hence we can conclude that \[\sigma | \rho \sim N(0, \frac{\rho}{1-\rho}\tau^2) \;, \; \rho \sim \text{Beta}(1/2, 1/2)\] is equivalent to \[\sigma |\lambda \sim N(0, \lambda^2 \tau^2)\;, \; \lambda\sim C^+(0,1)\]
    
    Now we shall derive tight bound for marginal prior density of $\sigma$
    \begin{align*}
        \pi^u(\sigma) &= \int_0^\infty p(\sigma, \lambda) \, d\lambda \\
        &= \int_0^\infty \frac{1}{\sqrt{2\pi \lambda^2 \tau^2}}\exp\Big(-\frac{1}{2\lambda^2 \tau^2}\sigma^2 \Big) \frac{2}{\pi} \frac{1}{(\lambda^2+1)} \, d\lambda \\
        & \text{change of variable} : u = 1 / \lambda^2 \Leftrightarrow \lambda = u^{-1/2} \quad d\lambda = -\frac{1}{2} u^{-3/2}\, du \\
        &= \int_0^\infty \frac{1}{\sqrt{2\pi \tau^2}} u^{1/2}\exp\Big(-\frac{1}{2\tau^2}\sigma^2 u \Big) \frac{2}{\pi}\frac{u}{1+u} \frac{1}{2}u^{-3/2}\, du \\
        &= \int_0^\infty \frac{1}{\sqrt{2\pi^3 \tau^2}} \exp\Big(-\frac{\sigma^2}{2\tau^2} u \Big) \frac{1}{1+u} \, du \\
        & \text{change of variable} : z = 1+u \Leftrightarrow u = z-1 \\
        &= \frac{1}{\tau \sqrt{2\pi^3}} \exp\Big(\frac{\sigma^2}{2\tau^2} \Big) \int_1^\infty \frac{1}{z} \exp\Big(-\frac{\sigma^2}{2\tau^2}z \Big)\, dz
    \end{align*}

    Define exponential integral $E_1$ as the following : \[E_1(x) = \int_1^\infty \frac{1}{z}\exp(-zx)\, dz \quad \forall \; x>0  \] Then it has tight bound given as \[\frac{1}{2}\exp(-x)\log\Big( 1+ \frac{2}{x}\Big) < E_1(x) < \exp(-x)\log\Big(1+\frac{1}{x} \Big) \quad x>0 \] 
    Note that this tight bound is mentioned in \href{https://en.wikipedia.org/wiki/Exponentia\_integral}{Wikipedia : Exponential integral}

    Thus, we have 
    \begin{align*}
        \pi^u(\sigma) &= \frac{1}{\tau \sqrt{2\pi^3}} \exp\Big(\frac{\sigma^2}{2\tau^2} \Big) E_1\Big(\frac{\sigma^2}{2\tau^2} \Big)
    \end{align*}
    Using the tight bound of $E_1$ given above, we get
    \begin{align*}
        \pi^u(\sigma) &< \frac{1}{\tau \sqrt{2\pi^3}}\log\Big(1+\frac{2\tau^2}{\sigma^2} \Big) \\
        \pi^u(\sigma) &> \frac{1}{2\tau \sqrt{2\pi^3}}\log\Big(1+\frac{4\tau^2}{\sigma^2} \Big) 
    \end{align*}
    From now on, we will call these two inequalities as upper and lower bound of marginal prior density of $\sigma_{ij}$ respectively. 

    Then, using lower bound of marginal prior density of $\sigma_{ij}$, we have the following :
    \begin{align*}
        \pi_{ij}^u(x) &\geq \frac{1}{2\tau}\sqrt{\frac{1}{2\pi^3}}\log\Big(1+\frac{4\tau^2}{x^2}\Big) \\ 
        &\geq \frac{1}{4\tau}\sqrt{\frac{1}{2\pi^3}}\frac{4\tau^2}{x^2} \quad \because \log(1+x)\geq \frac 12 x \quad \text{when} \; 0\leq x \leq 1 \quad \text{and} \quad \tau/x \; \text{is suff. small} \\
        &= \sqrt{\frac{1}{2\pi^3}}\frac{\tau}{x^2}
    \end{align*}
\end{proof}

\noindent \textbf{Proof of Theorem 5}
\begin{proof}
    Note that $B_{\eps}$ is defined as \[B_\eps = \{f_\Sigma : \Sigma \in \Cc_p , \; K(f_{\Sigma_0}, f_\Sigma)< \eps^2 , \; V(f_{\Sigma_0}, f_\Sigma)< \eps^2 \}\]

    By Lemma 3, it suffices to show that \[\pi\Big(\norm{\Sigma- \Sigma_0}_F^2 \leq \frac{2}{3\zeta^4\zeta_0^2}\eps_n^2\Big)\geq \exp(-Cn\eps_n^2)\] 

    This is because 
    \begin{align*}
        &\norm{\Sigma- \Sigma_0}_F^2 \leq \frac{2}{3\zeta^4\zeta_0^2}\eps_n^2 \\ 
        &\Rightarrow K(f_{\Sigma_0, f_\Sigma}) \leq \zeta^4\zeta_0^2 \norm{\Sigma - \Sigma_0}_F^2 \leq \frac 23 \eps_n^2 < \eps_n^2 \quad \text{and} \quad V(f_{\Sigma_0}, f_\Sigma) \leq \frac 32 \zeta^4\zeta_0^2 \norm{\Sigma - \Sigma_0}_F^2 \leq \eps_n^2 \\
        &\Rightarrow f_\Sigma \in B_{\eps_n}
    \end{align*}

    so that $\pi(B_{\eps_n}) \geq \pi\Big(\norm{\Sigma- \Sigma_0}_F^2 \leq \frac{2}{3\zeta^4\zeta_0^2}\eps_n^2\Big)$

    Note that 
    \begin{align*}
        &\pi\Big(\norm{\Sigma- \Sigma_0}_F^2 \leq \frac{2}{3\zeta^4\zeta_0^2}\eps_n^2\Big) = \pi\Big(\norm{\Sigma- \Sigma_0}_F^2 \leq \frac{2}{3\zeta^4\zeta_0^2} \frac{(p+s_0)\log p}{n} \Big) \\
        &\geq \pi\Big( \sum_{i\neq j}(\sigma_{ij} - \sigma_{ij}^*)^2 \leq \frac{2}{3\zeta^4\zeta_0^2}\frac{s_0\log p}{n} \;, \; \sum_{j=1}^p(\sigma_{jj} - \sigma_{jj}^*)^2 \leq \frac{2}{3\zeta^4\zeta_0^2}\frac{p\log p}{n} \Big) \quad \because \; x\leq \alpha, y \leq \gamma \Rightarrow x+y \leq \alpha + \gamma \\ 
        &\geq \pi\Big(\max_{i\neq j}(\sigma_{ij}- \sigma_{ij}^*)^2\leq \frac{2}{3\zeta^4\zeta_0^2}\frac{s_0\log p}{p(p-1)n} \;,\; \max_{1\leq j\leq p}(\sigma_{jj}- \sigma_{jj}^*)^2 \leq \frac{2}{3\zeta^4 \zeta_0^2}\frac{\log p}{n} \Big) := \pi(A_{n, \Sigma_0})
    \end{align*}
    where $\Sigma_0 = (\sigma_{ij}^*)$

    We will introduce Weyl's theorem here. (Source : \href{https://en.wikipedia.org/wiki/Weyl%27s_inequality}{Wikipedia : Weyl's inequality})

    If $A, B$ are $n\times n$ symmetric (or Hermitian) matrices then \[\lambda_k(A) + \lambda_n(B) \leq \lambda_k(A+B) \leq \lambda_k(A) +\lambda_1(B)\quad \forall\; k=1, \cdots, n\] where $\lambda_1(M)\geq \cdots \geq \lambda_n(M)$ are eigenvalues of symmetric matrix $M\in \Real^{n\times n}$

    Here, we will plug in $A= \Sigma_0$ , $B= \Sigma - \Sigma_0$ so that $A+B = \Sigma$ 

    Also, we will use two more properties about matrix norm. 

    The first one is that for symmetric A, we have $-\norm{A}_2 \leq \lambda(A)\leq \norm{A}_2$ where $\lambda(A)$ is any eigenvalue of $A$. This is because \[\lambda(A)^2 = \lambda(A^2) = \lambda(A^T A) \Rightarrow \abs{\lambda(A)} = \sqrt{\lambda(A^T A)}\leq \norm{A}_2 \]

    The second one is the special case of the H\"{o}lder inequality $\norm{A}_2 \leq \sqrt{\norm{A}_1\norm{A}_\infty}$ (Source : \href{https://en.wikipedia.org/wiki/Matrix_norm}{Wikipedia : Matrix Norm}) Also, if $A$ is symmetric, then $\norm{A}_1 = \norm{A}_\infty$ since the former is maximum absolute column sum and the latter is maximum absolute row sum. Thus we get $\norm{A}_2 \leq \norm{A}_1$ given $A$ is symmetric.

    We want to show that \[\Sigma \in A_{n, \Sigma_0} \Rightarrow \Sigma \in \Uc(\zeta)\]

    Suppose $\Sigma \in A_{n, \Sigma_0}$. Then we have \[\norm{\Sigma- \Sigma_0}_1 \leq (p-1)\max_{i\neq j}\abs{\sigma_{ij}- \sigma_{ij}^*} + \max_{1\leq j\leq p}\abs{\sigma_{jj} - \sigma_{jj}^*}\]
    \begin{align*}
        \lambda_{min}(\Sigma) &\geq \lambda_{min}(\Sigma_0) + \lambda_{min}(\Sigma-\Sigma_0) \quad \because \; \text{Weyl's thm} \\ 
        &\geq \lambda_{min}(\Sigma_0) - \norm{\Sigma- \Sigma_0}_2 \quad \because \; -\norm{A}_2 \leq \lambda(A) \leq \norm{A}_2 \\
        &\geq \lambda_{min}(\Sigma_0) - \norm{\Sigma-\Sigma_0}_1 \quad \because \; \norm{A}_2 \leq \norm{A}_1 \; \text{given } A \text{ is symmetric } \\
        &\geq \zeta_0^{-1} - \Big\{ (p-1)\sqrt{ \frac{2}{3\zeta^4\zeta_0^2}\frac{s_0\log p}{p(p-1)n}}  + \sqrt{\frac{2}{3\zeta^4 \zeta_0^2}\frac{\log p}{n} } \Big\} \quad \because \; \Sigma_0 \in \Uc(s_0, \zeta_0)\;,\; \Sigma \in A_{n, \Sigma_0} \\
        &:= \zeta_0^{-1} - \star \\
        & \\
        \lambda_{max}(\Sigma) &\leq \lambda_{max}(\Sigma_0) + \lambda_{max}(\Sigma- \Sigma_0) \\
        &\leq \lambda_{max}(\Sigma_0) + \norm{\Sigma - \Sigma_0}_2 \\
        &\leq \lambda_{max}(\Sigma_0) + \norm{\Sigma - \Sigma_0}_1 \\
        &\leq \zeta_0 + \Big\{ (p-1)\sqrt{ \frac{2}{3\zeta^4\zeta_0^2}\frac{s_0\log p}{p(p-1)n}}  + \sqrt{\frac{2}{3\zeta^4 \zeta_0^2}\frac{\log p}{n} } \Big\} \\
        &= \zeta_0 + \star 
    \end{align*}
  

    We shall claim that $\star \rightarrow 0$ as $n\rightarrow \infty$

    \[\star \leq \sqrt{\frac{2}{3\zeta^4\zeta_0^2}} \sqrt{\frac{(s_0+1)\log p}{n}} \leq \sqrt{\frac{2}{3\zeta^4\zeta_0^2}} \sqrt{\frac{(s_0+p)\log p}{n}} \rightarrow 0 \quad \because \; \eps_n \rightarrow 0 \]

    Thus, combining the fact that $\zeta_0 < \zeta$ and $\star \rightarrow 0$, we get \[\lambda_{min}(\Sigma)\geq \zeta_0^{-1} - \star > \zeta^{-1} \quad \text{and} \quad \lambda_{max}(\Sigma)\leq \zeta_0 + \star < \zeta \quad \text{for all suff. large } n\]

    Hence, we have shown that \[\Sigma \in A_{n, \Sigma_0} \Rightarrow \Sigma \in \Uc(\zeta)\] as desired.

    Using above, we get $\pi(A_{n, \Sigma_0})\geq \pi^u(A_{n, \Sigma_0})$ since \[\pi(A_{n, \Sigma_0}) = \frac{\pi^u(A_{n, \Sigma_0})\text{I}(\Sigma\in \Uc(\zeta))}{\pi^u(\Sigma\in \Uc(\zeta))} = \frac{\pi^u(A_{n, \Sigma_0})}{\pi^u(\Sigma\in \Uc(\zeta))} \geq \pi^u(A_{n, \Sigma_0}) \quad \because \; \pi^u(\Sigma \in \Uc(\zeta))\leq 1\]

    Here, we shall briefly check what we have already shown. 
    \[\pi(B_{\eps_n}) \geq \pi\Big(\norm{\Sigma- \Sigma_0}_F^2 \leq \frac{2}{3\zeta^4\zeta_0^2}\eps_n^2\Big) \geq \pi(A_{n, \Sigma_0})\geq \pi^u(A_{n, \Sigma_0}) \]

    Hence, from now on, our goal is to prove that \[\pi^u(A_{n, \Sigma_0})\geq \exp(-Cn\eps_n^2)\]

    Note that
    \begin{align*}
        \pi^u(A_{n, \Sigma_0}) &= \pi^u\Big(\max_{i\neq j}(\sigma_{ij}- \sigma_{ij}^*)^2\leq \frac{2}{3\zeta^4\zeta_0^2}\frac{s_0\log p}{p(p-1)n} \;,\; \max_{1\leq j\leq p}(\sigma_{jj}- \sigma_{jj}^*)^2 \leq \frac{2}{3\zeta^4 \zeta_0^2}\frac{\log p}{n}\Big) \\
        &= \pi^u\Big(\max_{i\neq j}(\sigma_{ij}- \sigma_{ij}^*)^2\leq \frac{2}{3\zeta^4\zeta_0^2}\frac{s_0\log p}{p(p-1)n}\Big) \times \pi^u\Big(\max_{1\leq j\leq p}(\sigma_{jj}- \sigma_{jj}^*)^2 \leq \frac{2}{3\zeta^4 \zeta_0^2}\frac{\log p}{n} \Big) \\
        &= \prod_{i < j} \pi^u\Big((\sigma_{ij}- \sigma_{ij}^*)^2\leq \frac{2}{3\zeta^4\zeta_0^2}\frac{s_0\log p}{p(p-1)n} \Big) \times \prod_{j=1}^p \pi^u\Big((\sigma_{jj}- \sigma_{jj}^*)^2 \leq \frac{2}{3\zeta^4 \zeta_0^2}\frac{\log p}{n} \Big)
    \end{align*}
    This is because all elements of $\Sigma$ are independent to each other given unconstrained setting. 

    Observe $\prod_{j=1}^p \pi^u\Big((\sigma_{jj}- \sigma_{jj}^*)^2 \leq \frac{2}{3\zeta^4 \zeta_0^2}\frac{\log p}{n} \Big)$ first. We want to find a lower bound of this term.
    \begin{align*}
        &\prod_{j=1}^p \pi^u\Big((\sigma_{jj}- \sigma_{jj}^*)^2 \leq \frac{2}{3\zeta^4 \zeta_0^2}\frac{\log p}{n} \Big) = \prod_{j=1}^p \pi^u\Big(\abs{\sigma_{jj}- \sigma_{jj}^*} \leq \sqrt\psi \Big) \quad \text{where} \; \psi:= \frac{2}{3\zeta^4 \zeta_0^2}\frac{\log p}{n} \\
        &= \prod_{j=1}^p \pi^u(\sigma_{jj}^* -\sqrt \psi \leq \sigma_{jj}\leq \sigma_{jj}^* + \sqrt \psi ) \quad \because \; \psi \rightarrow 0 \; \text{so that}\; \sigma_{jj}^* - \sqrt \psi \geq 0 \; \text{for all suff. large } n \\ 
        &\text{Note that } \sigma_{jj}\sim \Gamma(1, \lambda/2)\; \text{and}\; \pi^u(\sigma_{jj}) = \frac{\lambda}{2}\exp(-\frac{\lambda}{2}\sigma_{jj}) \; \text{is decreasing function } \\
        &\geq \prod_{j=1}^p 2\sqrt\psi \,  \pi^u(\sigma_{jj}^* + \sqrt \psi) = \prod_{j=1}^p 2\sqrt\psi \frac \lambda 2 \exp(-\frac \lambda 2 (\sigma_{jj}^* + \sqrt \psi )) = \prod_{j=1}^p \sqrt\psi \lambda  \exp(-\frac \lambda 2 (\sigma_{jj}^* + \sqrt \psi )) \\
        &\geq \prod_{j=1}^p \sqrt\psi \lambda  \exp(-\frac \lambda 2 (\zeta_0 + \sqrt \psi )) \quad \because \; \sigma_{jj}^* \leq \lambda_{max}(\Sigma_0)\leq \zeta_0 \quad \text{due to energy boundedness} \\
        &= \Big\{\sqrt\psi \lambda  \exp(-\frac \lambda 2 (\zeta_0 + \sqrt \psi ))  \Big\}^p
    \end{align*}
    Using a condition $\log p/\zeta^4 \zeta_0^4 \leq n$ , we have $\lambda\sqrt \psi \leq \lambda \zeta_0$ since \[\lambda \sqrt{\psi} = \lambda \sqrt{\frac{2}{3\zeta^4 \zeta_0^2}\frac{\log p}{n}} = \lambda \zeta_0 \sqrt{\frac{2}{3\zeta^4 \zeta_0^4}\frac{\log p}{n}} \leq \lambda \zeta_0 \sqrt{\frac 23}\leq \lambda \zeta_0\]

    Hence, we can proceed the above inequality as the following :
    \begin{align*}
        &\prod_{j=1}^p \pi^u\Big((\sigma_{jj}- \sigma_{jj}^*)^2 \leq \frac{2}{3\zeta^4 \zeta_0^2}\frac{\log p}{n} \Big) \geq \Big\{\sqrt\psi \lambda  \exp(-\frac \lambda 2 (\zeta_0 + \sqrt \psi ))  \Big\}^p \\
        &= \exp(p\log \lambda \sqrt \psi)\exp\Big(- \frac \lambda 2 p \zeta_0 - \frac \lambda 2 p \sqrt \psi \Big) \\
        &\geq \exp(p\log \lambda \sqrt \psi)\exp\Big(- \frac \lambda 2 p \zeta_0 - \frac \lambda 2 p \zeta_0 \Big)  \quad \because \; \lambda\sqrt \psi \leq \lambda \zeta_0 \\
        &= \exp\Big(- p\lambda \zeta_0 - p \log \frac{1}{\lambda\sqrt{\psi}}\Big) \\
        &\geq \exp\Big(-p \log p - p \log \frac{1}{\lambda\sqrt{\psi}} \Big) \quad \because \; \lambda < \log p / \zeta_0 \; \text{by assumption}
    \end{align*}

    Here, we shall claim that $1/\sqrt \psi \leq \zeta^3 p^{1/2\beta}$ for all sufficiently large $n$

    \begin{align*}
        1/\sqrt \psi &= \sqrt{\frac 32}\zeta_0 \zeta^2 \sqrt{\frac{n}{\log p}} \\
        &< \sqrt{\frac 32}\zeta^3 \sqrt{\frac{n}{\log p}} \quad \because \zeta_0 < \zeta \; \text{by assumption} \\ 
        &\leq \sqrt{\frac 32}\zeta^3 Cp^{1/2\beta} \frac{1}{\sqrt{\log p}} \quad \because \; p \asymp n^\beta \,,\, n^\beta\leq Cp \text{ for some } C>0 \text{ by assumption } \\
        &\leq \zeta^3p^{1/2\beta} \quad \because \; p \; \text{gets large enough to attain} \; \sqrt{3/2}C/\sqrt{\log p}<1
    \end{align*}

    We will complete our process of finding lower bound of $\prod_{j=1}^p \pi^u\Big((\sigma_{jj}- \sigma_{jj}^*)^2 \leq \frac{2}{3\zeta^4 \zeta_0^2}\frac{\log p}{n} \Big)$ as the below. 
    \begin{align*}
        &\prod_{j=1}^p \pi^u\Big((\sigma_{jj}- \sigma_{jj}^*)^2 \leq \frac{2}{3\zeta^4 \zeta_0^2}\frac{\log p}{n} \Big) \geq \exp\Big(-p \log p - p \log \frac{1}{\lambda\sqrt{\psi}} \Big) \\
        &\geq \exp\Big(-p\log p - p\log \frac{\zeta^3 p^{1/2\beta}}{\lambda} \Big) \quad \because \; 1/\sqrt \psi \leq \zeta^3 p^{1/2\beta} \; \text{for all sufficiently large } n \\
        &\geq \exp\Big(-p \log p - p (1 + \frac 34 + \frac{1}{2\beta})\log p\Big) \quad \because \; p^{-1}<\lambda \; \text{and} \; \zeta^4\leq p \quad \text{by assumption} \\
        &\geq \exp\Big(-(3+\frac{1}{2\beta})p \log p\Big)
    \end{align*}

    Hence we have
    \begin{equation} \label{diagonal ineq}
        \prod_{j=1}^p \pi^u\Big((\sigma_{jj}- \sigma_{jj}^*)^2 \leq \frac{2}{3\zeta^4 \zeta_0^2}\frac{\log p}{n} \Big) \geq \exp\Big(-(3+\frac{1}{2\beta})p \log p\Big)
    \end{equation}
    for sufficiently large $n$.

    Next, we shall find a lower bound of $\prod_{i < j} \pi^u\Big((\sigma_{ij}- \sigma_{ij}^*)^2\leq \frac{2}{3\zeta^4\zeta_0^2}\frac{s_0\log p}{p(p-1)n} \Big)$. Note that it can be decomposed as the following.
    
    \begin{align*}
        &\prod_{i < j} \pi^u\Big((\sigma_{ij}- \sigma_{ij}^*)^2\leq \frac{2}{3\zeta^4\zeta_0^2}\frac{s_0\log p}{p(p-1)n} \Big) = \prod_{i<j} \pi^u(\abs{\sigma_{ij} - \sigma_{ij}^*}\leq \sqrt \phi)  \quad \text{where} \; \phi = \frac{2}{3\zeta^4\zeta_0^2}\frac{s_0\log p}{p(p-1)n} \\
        &= \prod_{(i,j)\in s(\Sigma_0)}\pi^u(\abs{\sigma_{ij} - \sigma_{ij}^*}\leq \sqrt \phi) \times \prod_{(i,j) \notin s(\Sigma_0)\, ,\, i<j}\pi^u(\abs{\sigma_{ij}}\leq \sqrt \phi) 
    \end{align*}

    Before finding the lower bound of those two terms above, recall the tight bound of marginal prior density of off diagonal $\sigma_{ij}$ of covariance matrix.
    \begin{align*}
        \pi^u(\sigma_{ij}) &< \frac{1}{\tau \sqrt{2\pi^3}}\log\Big(1+\frac{2\tau^2}{\sigma_{ij}^2} \Big) \\
        \pi^u(\sigma_{ij}) &> \frac{1}{2\tau \sqrt{2\pi^3}}\log\Big(1+\frac{4\tau^2}{\sigma_{ij}^2} \Big) 
    \end{align*}
    
    We will deal with $\prod_{(i,j) \notin s(\Sigma_0)\, ,\, i<j}\pi^u(\abs{\sigma_{ij}}\leq \sqrt \phi)$ first. 

    \begin{align*}
        &\pi^u(\abs{\sigma_{ij}}>\sqrt \phi) = 2\pi^u(\sigma_{ij}>\sqrt \phi) = \int_{\sqrt\phi}^\infty \pi^u(\sigma_{ij})\, d\sigma_{ij}\\
        &\leq 2\int_{\sqrt\phi}^\infty \frac{1}{\tau \sqrt{2\pi^3}}\log\Big(1+\frac{2\tau^2}{\sigma_{ij}^2} \Big) \,d\sigma_{ij} \quad \because \; \text{upper bound of marginal prior density of $\sigma_{ij}$} \\
        &\leq 2\int_{\sqrt \phi}^\infty \frac{1}{\tau \sqrt{2\pi^3}} \frac{2\tau^2}{\sigma_{ij}^2} \, d\sigma_{ij} \quad \because \; \log(1+x)\leq x \quad \forall \; x>-1 \quad \text{by supporting line lemma}\\
        &= 2\tau \sqrt{\frac{2}{\pi^3}} \int_{\sqrt{\phi}}^\infty \frac{1}{\sigma_{ij}^2}\, d\sigma_{ij} = 2\tau \sqrt{\frac{2}{\pi^3}} \frac{1}{\sqrt{\phi}} \\
        &\\
        &\prod_{(i,j) \notin s(\Sigma_0)\, ,\, i<j}\pi^u(\abs{\sigma_{ij}}\leq \sqrt \phi) = \prod_{(i,j) \notin s(\Sigma_0)\, ,\, i<j} \big(1 - \pi^u(\abs{\sigma_{ij}}>\sqrt{\phi})\big) \\
        &\geq \prod_{(i,j) \notin s(\Sigma_0)\, ,\, i<j} \Big(1-  2\tau \sqrt{\frac{2}{\pi^3}} \frac{1}{\sqrt{\phi}} \Big) \quad \because \; \pi^u(\abs{\sigma_{ij}}>\sqrt \phi) \leq 2\tau \sqrt{\frac{2}{\pi^3}} \frac{1}{\sqrt{\phi}}  \\
        &\geq \Big(1-  2\tau \sqrt{\frac{2}{\pi^3}} \frac{1}{\sqrt{\phi}} \Big)^{p^2} \\
        &\geq \exp\Big(-4\tau \sqrt{\frac{2}{\pi^3}} \frac{1}{\sqrt{\phi}} \Big)^{p^2} \quad \because \; \log(1-x)\geq -2x \quad \text{when}\; 0\leq x\leq 1/2 \\
        &\text{Note that} \; \tau/\sqrt{\phi} \; \text{is small enough when $n$ is sufficiently large} \quad \because\; (p^2\sqrt{n})^{-1}\lesssim \tau \lesssim (p^2\sqrt{n})^{-1}\sqrt{s_0\log p} \\
        &\tau/\sqrt{\phi} \leq C \frac{1}{p^2}\sqrt{\frac{s_0\log p}{n}}\sqrt{\frac{p(p-1)n}{s_0\log p}}\frac{2}{3\zeta_0^2\zeta^4} \leq \tilde C \frac{1}{p} \rightarrow 0
    \end{align*}

    We can proceed the inequality as the following.
    \begin{align*}
        &\prod_{(i,j) \notin s(\Sigma_0)\, ,\, i<j}\pi^u(\abs{\sigma_{ij}}\leq \sqrt \phi) \geq \exp\Big(-4\tau \sqrt{\frac{2}{\pi^3}} \frac{1}{\sqrt{\phi}} \Big)^{p^2} \\
        &= \exp\Big(-4\tau p^2 \sqrt{\frac{2}{\pi^3}} \frac{1}{\sqrt{\phi}} \Big) \\
        &\geq \exp\Big(-4\sqrt{\frac{2}{\pi^3}}\tilde C p^2\frac 1p \Big) \quad \because \; \tau/\sqrt{\phi}\leq \tilde C \frac 1p \quad \text{by above} \\
        &= \exp(-Cp) \quad \text{for some } C>0
    \end{align*}

    Thus we have
    \begin{equation} \label{offdiagonal zeros ineq}
        \prod_{(i,j) \notin s(\Sigma_0)\, ,\, i<j}\pi^u(\abs{\sigma_{ij}}\leq \sqrt \phi) \geq \exp\Big(-4\tau p^2 \sqrt{\frac{2}{\pi^3}} \frac{1}{\sqrt{\phi}} \Big) \geq \exp(- Cp)
    \end{equation}
    for sufficiently large $n$.

    Finally, we shall find the lower bound of $\prod_{(i,j)\in s(\Sigma_0)}\pi^u(\abs{\sigma_{ij} - \sigma_{ij}^*}\leq \sqrt \phi)$

    Recall that marginal prior density of off diagonal $\sigma_{ij}$ given as $\pi^u(\sigma_{ij})$ is decreasing function with respect to $\abs{\sigma_{ij}}$ since \[ \pi^u(\sigma_{ij})= \int_0^\infty \frac{1}{\sqrt{2\pi^3 \tau^2}} \exp\Big(-\frac{\sigma_{ij}^2}{2\tau^2} u \Big) \frac{1}{1+u} \, du \] 
    Also, note that since $\phi = \frac{2}{3\zeta^4\zeta_0^2}\frac{s_0\log p}{p(p-1)n}\rightarrow 0 $ as $n$ tends to sufficiently large, we can write 
    \begin{equation*}
        (\abs{\sigma_{ij}- \sigma_{ij}^*}\leq \sqrt\phi) = (\sigma_{ij}^* - \sqrt\phi \leq \sigma_{ij} \leq \sigma_{ij}^* + \sqrt \phi) \begin{cases}\subset (0, \infty) & \; \text{if} \quad \sigma_{ij}^* >0 \\
        \subset (-\infty, 0) &\; \text{if} \quad \sigma_{ij}^* <0
        \end{cases}
    \end{equation*}
    for sufficiently large $n$, since $\sigma_{ij}^* \neq 0 \; \text{due to} \;  (i, j)\in s(\Sigma_0)$

    Therefore, we have the following inequality.
    \begin{align*}
       \pi^u(\abs{\sigma_{ij} - \sigma_{ij}^*}\leq \sqrt \phi) \begin{cases}
        \geq 2\sqrt{\phi} \, \pi^u(\sigma_{ij}^* + \sqrt{\phi}) & \quad \text{if} \quad \sigma_{ij}^* >0 \\
        \geq 2\sqrt{\phi} \, \pi^u(\sigma_{ij}^* - \sqrt{\phi}) & \quad \text{if} \quad \sigma_{ij}^* <0 
        \end{cases}
    \end{align*}

    Combining three facts, we can yield $\abs{\sigma_{ij}^*} \leq \zeta_0$ for all $i\neq j$. Those facts are given as the following.
    \begin{enumerate}
        \item The largest entry in magnitude of positive definite matrix lies on the diagonal (Source : Gockenbagh Linear Algebra Lemma 386)
        \item Energy boundedness : $\lambda_n \norm{x}_2^2 \leq x^T A x \leq \lambda_1 \norm{x}_2^2$ if symmetric $A$ has $\text{spec}(A) = \{\lambda_1\geq \cdots \geq \lambda_n\}$
        \item $\lambda_{max}(\Sigma_0)\leq \zeta_0$ by $\Sigma_0 \in \Uc(s_0, \zeta_0)$ assumption
    \end{enumerate}

    Hence we have
    \begin{align*}
        \abs{\sigma_{ij}^*} \leq \max_k{\abs{\sigma_{kk}^*}} = \max_k \sigma_{kk}^* \leq \lambda_{max}(\Sigma_0) \leq \zeta_0
    \end{align*}
    and what follow this are 
    \begin{align*}
        &\sigma_{ij}^* + \sqrt{\phi} \leq 2\zeta_0 \quad \text{if} \quad \sigma_{ij}^* >0 \\
        &\abs{\sigma_{ij}^* - \sqrt{\phi}} \leq 2\zeta_0 \quad \text{if} \quad \sigma_{ij}^* < 0
    \end{align*}

    Using this, we get \[\pi^u(\abs{\sigma_{ij} - \sigma_{ij}^*}\leq \sqrt \phi) \geq 2\sqrt{\phi}\, \pi^u(2\zeta_0)\]

    \begin{align*}
        &\prod_{(i,j)\in s(\Sigma_0)}\pi^u(\abs{\sigma_{ij} - \sigma_{ij}^*}\leq \sqrt \phi) \geq \Big(2\sqrt{\phi}\, \pi^u(2\zeta_0) \Big)^{s_0} \geq \Big(\sqrt{\phi}\, \pi^u(2\zeta_0) \Big)^{s_0} \geq \Bigg(\pi^u(2\zeta_0)\sqrt{\frac{2s_0\log p}{3\zeta^4 \zeta_0^2 p^2 n}} \Bigg)^{s_0} \\
        &= \exp\Bigg(s_0\log \pi^u(2\zeta_0) + \frac 12 s_0 \log \frac{2s_0\log p}{3\zeta^4\zeta_0^2p^2 n} \Bigg) \\
        &\geq \exp\Big(s_0 \log \pi^u(2\zeta_0) + \frac 12 s_0 \log \frac 23\frac{1}{\zeta^2p^2n} \Big) \quad \because\; \zeta^2 \zeta_0^2 \leq s_0\log p \quad \text{by assumption}
    \end{align*}

    Note that by taking advantage of Lemma 4, we can write
    \[\pi^u(2\zeta_0)\geq \frac{1}{\sqrt{2\pi^3}}\frac{\tau}{4\zeta_0^2} \]
    Of course, we should show that $\tau/2\zeta_0$ is sufficiently small to justify the use of Lemma 4. Since $\zeta_0$ is fixed, we shall show that $\tau \rightarrow 0$ as $n\rightarrow \infty$

    \begin{align*}
        \tau & \lesssim \frac{\sqrt{s_0\log p}}{p^2\sqrt{n}}  \quad \text{by assumption} \\
        & \lesssim \frac{\sqrt{s_0\log s_0}}{p^2\sqrt{n}} \quad \because p \lesssim s_0 \\ 
        &\leq  \frac{s_0}{p^2\sqrt{n}} \leq \frac{1}{\sqrt{n}} \quad \because \; \log s_0 \leq s_0 \leq p^2 
    \end{align*}

    Thus $\tau \lesssim \frac{1}{\sqrt{n}}$ so that $\tau / \zeta_0$ is sufficiently small as $n$ gets sufficiently large.

    Combining with the condition $\tau \gtrsim 1/\sqrt{n}p^2$ , we get 
    \[\pi^u(2\zeta_0)\geq \frac{1}{\sqrt{2\pi^3}}\frac{\tau}{4\zeta_0^2} \gtrsim \frac{1}{\sqrt{n}p^2} \]

    Now we shall proceed our target inequality.

    \begin{align*}
        &\prod_{(i,j)\in s(\Sigma_0)}\pi^u(\abs{\sigma_{ij} - \sigma_{ij}^*}\leq \sqrt \phi)\\
        &\geq \exp\Big(s_0 \log \pi^u(2\zeta_0) + \frac 12 s_0 \log \frac 23\frac{1}{\zeta^2p^2n} \Big)  \\
        &\geq \exp\Big(s_0 \log \Big(\frac{\tilde C}{\sqrt{n}p^2} \Big) + \frac 12 s_0 \log \frac 23\frac{1}{\zeta^2p^2n} \Big) \\
        &= \exp\Big(\frac 12 s_0 \log \Big(\frac{\tilde C^2}{np^4} \Big) + \frac 12 s_0 \log \frac 23\frac{1}{\zeta^2p^2n} \Big) \\
        &= \exp \Big(\frac 12 s_0 \log \big(\frac{2\tilde{C}^2}{3\zeta^2p^6n^2} \big)\Big)
    \end{align*}

    Here, we gonna use two inequalities
    \begin{enumerate}
        \item $C^* p^{-2/\beta} \leq n^{-2}$ for some $C^*>0$
        \item $1/\zeta^2 \geq 1/p$
    \end{enumerate}

    The first one comes from $p\asymp n^\beta$ so that $n^\beta \leq \tilde C^*p$ and the second one comes from $\zeta^4\leq p$ and $1<\zeta_0 < \zeta$. Using those inequalities, we get

    \begin{align*}
        &\prod_{(i,j)\in s(\Sigma_0)}\pi^u(\abs{\sigma_{ij} - \sigma_{ij}^*}\leq \sqrt \phi)\\
        &\geq \exp \Big(\frac 12 s_0 \log \big(\frac{2\tilde{C}^2}{3\zeta^2p^6n^2} \big)\Big) \\
        &\geq \exp\Big(\frac 12 s_0 \log\big(\frac 23 \tilde C^2 p^{-1}p^{-6} C^* p^{-2/\beta} \big) \Big) \\
        &= \exp\Big(\frac 12 s_0 \log (Cp^{-(7+\frac{2}{\beta})}) \Big) \\
        &= \exp\Big(-\frac12(7+\frac 2\beta)s_0 \log p + \frac 12 s_0 \log C \Big) \\
        &\geq \exp\Big(-\frac12(7+\frac 2\beta)s_0 \log p - \frac 12 s_0 \log p \Big) \quad \text{for suff. large } n \\
        &= \exp\Big(-(4+\frac 1\beta)s_0\log p \Big)
    \end{align*}

    Hence we have
    \begin{equation} \label{offdiagonal nonzeros ineq}
        \prod_{(i,j)\in s(\Sigma_0)}\pi^u(\abs{\sigma_{ij} - \sigma_{ij}^*}\leq \sqrt \phi) \geq \exp\Big(-(4+\frac 1\beta)s_0\log p \Big)  
    \end{equation}

    At last, combining \eqref{diagonal ineq}, \eqref{offdiagonal zeros ineq}, and \eqref{offdiagonal nonzeros ineq}, we have
    \begin{align*}
        &\pi^u(A_{n, \Sigma_0}) \geq \exp\Big(-(3+\frac{1}{2\beta})p \log p\Big) \times \exp(-Cp) \times \exp\Big(-(4+\frac 1\beta)s_0\log p \Big)  \\
        &\geq \exp\Big(-(3+\frac{1}{2\beta})p \log p\Big) \times \exp(-Cp) \times \exp\Big(-(4+\frac 1\beta)s_0\log p \Big) \times \exp(Cp) \times \exp(-p\log p) \\
        &= \exp\Big(-(4+\frac{1}{2\beta})p \log p\Big) \times \exp\Big(-(4+\frac 1\beta)s_0\log p \Big) \geq \exp\Big(-(4+\frac 1\beta)(p+s_0)\log p\Big) \\
        &= \exp\Big(-(4+\frac 1\beta)n\eps_n^2 \Big)
    \end{align*}

    Since we have already shown that \[\pi(B_{\eps_n}) \geq \pi\Big(\norm{\Sigma- \Sigma_0}_F^2 \leq \frac{2}{3\zeta^4\zeta_0^2}\eps_n^2\Big) \geq \pi(A_{n, \Sigma_0})\geq \pi^u(A_{n, \Sigma_0}) \] we can conclude that 
    \[\pi(B_{\eps_n}) \geq \exp\Big(-(4+\frac 1\beta) n\eps_n^2 \Big) \]


\end{proof}



\section{이경원, 정진욱, 김성민 - Theorem 2}

% 이경원
\input{contents/theorem2-kwlee.tex}

% 정진욱

이제 어떤 상수 $c_1>0$가 존재하여 모든 $i=1, \dots, r$에 대해
\[
\|\overline{\mathbb{P}}_{i,0}\wedge \overline{\mathbb{P}}_{i,1}\| \ge c_1
\]
임을 보이면 되는데 여기서는 $\|\overline{\mathbb{P}}_{1,0}\wedge \overline{\mathbb{P}}_{1,1}\| \ge c_1$, 즉 $i=1$일 때 만을 보일 것이다. $i$가 바뀌어도 동일한 상수 $c_1$을 얻을 수 있기에 이것만으로 충분하다. \newline

먼저 다음과 같은 변수 공간들을 고려하자.

\[
\Lambda_1 := \{ \lambda_1 (\theta) \in \mathbb{R}^p \ : \ \theta\in\Theta\}, \quad \Lambda_{-1}:= \{ \lambda_{-1}(\theta) \equiv (\lambda_2(\theta), \dots, \lambda_r(\theta))^T \in \mathbb{R}^{(r-1)\times p} \ : \ \theta\in\Theta\}. 
\]
각 $a \in \{0,1\}$, $b\in\{0,1\}^{r-1}$, $c\in \Lambda_{-1}$마다 다음과 같은 확률분포를 정의한다:
\[
\overline{\mathbb{P}}_{(1,a,b,c)} := \frac{1}{|\Theta_{(1,a,b,c)}|} \sum_{\theta \in \Theta_{(1,a,b,c)}} \mathbb{P}_\theta, \quad \Theta_{(1,a,b,c)} :=  \{\theta\in\Theta \ : \ \gamma_1(\theta) = a, \quad \gamma_{-1}(\theta) = b, \quad \lambda_{-1}(\theta) = c\}.
\]
여기서 $\gamma_{-1}(\theta) = (\gamma_2(\theta), \dots, \gamma_r(\theta))^T$ 이다. 이제 함수 $f= f(\gamma_{-1}, \lambda_{-1})$를 $\Theta_{-1}:= \{0,1\}^{r-1}\times\Lambda_{-1}$ 에서 평균을 취한 값을 $\mathbb{E}_{(\gamma_{-1}, \lambda_{-1})}f(\gamma_{-1}, \lambda_{-1})$라고 쓴다. 즉
\[
\mathbb{E}_{(\gamma_{-1}, \lambda_{-1})}f(\gamma_{-1}, \lambda_{-1}) := \frac{1}{2^{r-1}|\Lambda|} \sum_{(b,c)\in \Theta_{-1}} |\Theta_{(1,a,b,c)}| f(b,c)
\]
이고 $a$는 0과 1 중 무엇을 선택해도 무관하다.\\

이제 어떤 상수 $c_2\in(0,1)$가 존재하여

\begin{equation}\label{16}\tag{16}
\mathbb{E}_{(\gamma_{-1}, \lambda_{-1})}\left\{ \int \left(\frac{d\overline{\mathbb{P}}_{(1,1,\gamma_{-1}, \lambda_{-1})} }{d\overline{\mathbb{P}}_{(1,0,\gamma_{-1}, \lambda_{-1})} } \right)^2 d\overline{\mathbb{P}}_{(1,0,\gamma_{-1}, \lambda_{-1})} -1 \right\} \le c_2^2
\end{equation}
임을 보이기만 하면 충분하다. 왜냐하면 \cite[Lemma 8, (ii)]{cai2012optimal}의 결과를 이용하면 위 식은
\[
\| \overline{\mathbb{P}}_{1,0}\wedge \overline{\mathbb{P}}_{1,1} \| \ge 1-c_2 >0
\]
을 의미하기 때문이다. \\

\begin{quote}
여기서 잠시 이에 대한 증명을 짚고 넘어가자면, 우선 공통의 dominating measure $\mu$에 대해 두 개의 density $q_0$와 $q_1$가 있다고 할 때 옌센 부등식을 이용하면

\[
\left[ \int |q_0-q_1| d\mu\right]^2 = \left( \int \left|\frac{q_0 - q_1}{q_1} \right| q_1 \right)^2 \le \int \frac{(q_0 - q_1)^2}{q_1} d\mu   = \int \left(\frac{q_0^2}{q_1} -1 \right)d\mu
\]
이다. 위 식과 \eqref{16}을 이용하면 알 수 있는 사실은

\[
\mathbb{E}_{(\gamma_{-1}, \lambda_{-1})}\left\{ \int \left| d\overline{\mathbb{P}}_{(1,1,\gamma_{-1}, \lambda_{-1})} - d\overline{\mathbb{P}}_{(1,0,\gamma_{-1}, \lambda_{-1})}  \right|^2 \right\} \le c_2^2
\]
이고 코시-슈바르츠 부등식을 이용하면 

\[
\mathbb{E}_{(\gamma_{-1}, \lambda_{-1})}\left\{ \int \left| d\overline{\mathbb{P}}_{(1,1,\gamma_{-1}, \lambda_{-1})} - d\overline{\mathbb{P}}_{(1,0,\gamma_{-1}, \lambda_{-1})}  \right| \right\} \le c_2
\]
또한 성립함을 알 수 있다. 여기서 total variation affinity
\[
\| \mathbb{P} \wedge \mathbb{Q}\| = 1- TV(\mathbb{P} , \mathbb{Q})
\]
를 이용하면 결국
\[
\mathbb{E}_{(\gamma_{-1}, \lambda_{-1})}\left\{ \left\| \overline{\mathbb{P}}_{(1,1,\gamma_{-1}, \lambda_{-1})} \wedge \overline{\mathbb{P}}_{(1,0,\gamma_{-1}, \lambda_{-1})}  \right\| \right\} \ge 1- c_2
\]
를 얻게 된다. 마지막으로 $\overline{\mathbb{P}}_{1,0}$와  $\overline{\mathbb{P}}_{1,1}$가 각각 $\overline{\mathbb{P}}_{(1,0,\gamma_{-1}, \lambda_{-1})}$, $\overline{\mathbb{P}}_{(1,1,\gamma_{-1}, \lambda_{-1})}$와 같은 형태의 확률측도들에 대한 가중평균이라는 점을 기억한다면, \cite[Lemma 4]{cai2012optimal}를 이용하여


\[
\| \overline{\mathbb{P}}_{1,0} \wedge \overline{\mathbb{P}}_{1,1}\| \ge \mathbb{E}_{(\gamma_{-1}, \lambda_{-1})}\left\{ \left\| \overline{\mathbb{P}}_{(1,1,\gamma_{-1}, \lambda_{-1})} \wedge \overline{\mathbb{P}}_{(1,0,\gamma_{-1}, \lambda_{-1})}  \right\| \right\}
\]
를 얻게 되므로, 증명이 마무리된다. \\


\end{quote}


\vspace{0.4cm}

이제 \eqref{16}을 얻기 위해  $(\gamma_{-1}, \lambda_{-1})$가 고정되었을 때 \eqref{16}에서  등장하는 각각의 측도 $\overline{\mathbb{P}}_{(1,0,\gamma_{-1}, \lambda_{-1})} $ 와 $\overline{\mathbb{P}}_{(1,1,\gamma_{-1}, \lambda_{-1})} $이 어떠한 형태인지 알 필요가 있다. 먼저 $\overline{\mathbb{P}}_{(1,0,\gamma_{-1}, \lambda_{-1})}$의 경우 $\gamma_1=0$으로 설정되어 있는데, 정의를 다시 떠올려본다면
\[
\overline{\mathbb{P}}_{(1,0,\gamma_{-1}, \lambda_{-1})} = \frac{1}{|\Theta_{(1,0,\gamma_{-1}, \lambda_{-1})}|} \sum_{\theta \in \Theta_{(1,0,\gamma_{-1}, \lambda_{-1})}} \mathbb{P}_\theta
\]
이고 $\mathbb{P}_\theta$는 p차원 정규분포 $N_p(0,\Sigma(\theta))$를 따르는 $n$개의 표본 $X_1, \dots, X_n$에 대한 결합분포이다. 이때 $(\gamma_{-1}, \lambda_{-1})$는 고정되어 있으니, $\theta\in\Theta_{(1,0,\gamma_{-1}, \lambda_{-1})}$에 대응하는 모든 $\Sigma(\theta)$들은 
\[
\Sigma(\theta) = I_p + \epsilon_{np} \sum_{m=2}^r \gamma_m A_m (\lambda_m),
\]
즉 $\theta$에 대응하는 $\lambda_1(\theta)$가 무슨 값을 취하더라도 $\Sigma(\theta)$는 동일한 형태라는 것이다. 따라서 $\overline{\mathbb{P}}_{(1,0,\gamma_{-1}, \lambda_{-1})}$는 $p$차원 정규분포 $N_p(0,\Sigma_0)$를 따르는 $n$개의 표본에 대한 결합분포임을 알 수 있으며 $\Sigma_0$는 다음과 같이 주어진다:
\[
\Sigma_0 := \left(\begin{array}{c|c} 1 & 0_{1\times(p-1)} \\ \hline 
0_{(p-1)\times 1} & S_{(p-1)\times (p-1)}\end{array} \right).
\]
여기서 $S_{(p-1)\times(p-1)} = \{s_{ij}\}$은 $(p-1)\times (p-1)$ 대칭행렬로
\[
s_{ij} = \left\{\begin{array}{ll} 1 & i=j\\
\epsilon_{np} & \gamma_{i+1} =\lambda_{i+1, j+1} = 1\\
0 & \mbox{otherwise} \end{array}\right.
\]
와 같이 주어진다. 여기서 $i=1, \dots, p-r = p-\lfloor p/2 \rfloor$에 대해 $\lambda_{mi}=0$이므로 $A_m(\lambda_m)$은 대각성분을 가지지 않기에,  $\epsilon_{np}$는  $S_{(p-1)\times(p-1)}$의 대각 성분에 나타나지 않음을 유념하자. \\

이제 $\overline{\mathbb{P}}_{(1,1,\gamma_{-1}, \lambda_{-1})}$의 경우를 살펴보도록 하자. 우선 임의의 $c \in \Lambda_{-1}$에 대해
\[
\Lambda_1(c) := \{a \in \mathbb{R}^p \ : \ \lambda_1(\theta) = a, \quad \lambda_{-1}(\theta) =c \ \mbox{ for some } \ \theta \in \Theta \}
\]
를 정의하자. 또한 $\lambda_{-1} = (\lambda_2 (\theta), \dots, \lambda_r(\theta))^T = \left(\begin{array}{c}\lambda_2(\theta)^T\\ \hline \vdots \\ \hline \lambda_r(\theta)^T \end{array}\right) \in \mathbb{R}^{(r-1)\times p}$가 하나 주어졌을 때, $n_{\lambda_{-1}}$를 $\lambda_{-1}$의 각 열의 성분을 모두 더했을 때 $2k$가 되는 열의 갯수라고 하자. 즉
\[
n_{\lambda_{-1}} := \left| \left\{ i \ : \ \sum_{m=2}^r \lambda_{mi} = 2k\right\}\right|
\]
이고 $p_{\lambda_{-1}} = r -n_{\lambda_{-1}}$이라 두자. 이와 같이 정의하면 $p_{\lambda_{-1}}$은 $\lambda_1$의 성분 $\lambda_{1i}$ 중 0이 될 수도, 1이 될 수도 있는 성분의 갯수라는 것을 알 수 있다. 여기서 임의의 $\lambda_{-1}\in \Lambda_{-1}$에 대해 
\[
|\Lambda_1(\lambda_{-1}) | = \binom{p_{\lambda_{-1}}}{k}, \quad p_{\lambda_{-1}} \ge \frac p4 -1
\]
가 성립함을 기억하자. 위 식의 오른쪽은 $n_{\lambda_{-1}} \cdot 2k \le r k$, 즉 $n_{\lambda_{-1}} \le r/2$임을 이용하면 유도할 수 있다:
\[
p_{\lambda_{-1}} = r - n_{\lambda_{-1}} \ge \frac r2  \ge \frac p4 -1.
\]
부등식 $n_{\lambda_{-1}} \cdot 2k \le r k$이 성립한다는 것은 좌변은 행렬  $\lambda_{-1} =  \left(\begin{array}{c}\lambda_2(\theta)^T\\ \hline \vdots \\ \hline \lambda_r(\theta)^T \end{array}\right) $에서  $n_{\lambda_{-1}}$에 해당하는 열들만 성분을 더한 것이고 우변은 행렬 $\lambda =  \left(\begin{array}{c}\lambda_1(\theta)^T\\ \hline 
 \lambda_{-1}(\theta)^T \end{array}\right)$의 모든 성분을 더한 것이라는 것을 생각해보면 자명하다. 그러므로 $p$가 충분히 크면 $\Lambda_1(\lambda_{-1})$은 공집합이 아니다. 이제 $\overline{\mathbb{P}}_{(1,1,\gamma_{-1}, \lambda_{-1})}$의 정의를 다시 떠올려보면
\[
\overline{\mathbb{P}}_{(1,1,\gamma_{-1}, \lambda_{-1})} = \frac{1}{|\Theta_{(1,1,\gamma_{-1}, \lambda_{-1})}|} \sum_{\theta \in \Theta_{(1,1,\gamma_{-1}, \lambda_{-1})}} \mathbb{P}_\theta
\]
이고 $\mathbb{P}_\theta$는 p차원 정규분포 $N_p(0,\Sigma(\theta))$를 따르는 $n$개의 표본 $X_1, \dots, X_n$에 대한 결합분포이다. 이때 $\theta\in\Theta_{(1,1,\gamma_{-1}, \lambda_{-1})}$에 대응하는 $\Sigma(\theta)$들은 
\[
\Sigma(\theta) = I_p + \epsilon_{np} A_m(\lambda_1(\theta)) + \epsilon_{np} \sum_{m=2}^r \gamma_m A_m (\lambda_m)
\]
의 형태로, 앞선 경우와는 다르게 $\theta$에 대응하는 $\lambda_1(\theta)$가 바뀔 때 마다 $\Sigma(\theta)$가 바뀐다는 것을 알 수 있다. 고를 수 있는 $\lambda_1(\theta)$의 갯수는 총 $\binom{p_{\lambda_{-1}}}{k}$개가 있으므로, 따라서 $\overline{\mathbb{P}}_{(1,1,\gamma_{-1}, \lambda_{-1})}$는 평균이 0이고 공분산 행렬이 다음과 같은 형태인 $p$차원 정규분포를 따르는 $n$개의 표본에 대한 결합분포들 $\binom{p_{\lambda_{-1}}}{k}$개의 평균임을 알 수 있다:
\[
 \left(\begin{array}{c|c} 1 & r^T \\ \hline 
r & S_{(p-1)\times (p-1)}\end{array} \right).
\]
이때 $r\in \mathbb{R}^{p-1}$은 0이 아닌 성분의 갯수가 $k$개이며 0이 아닌 성분들은 모두 $\epsilon_{np}$이고, $S_{(p-1)\times(p-1)}$은 앞서 정의한 것과 같다. \\

 이제 \cite[p.2411]{cai2012optimal}에서 사용한 논증을 이용한다.우선 \cite[Lemma 9]{cai2012optimal}로부터 다음을 알 수 있다: 각 $i=0,1,2$에 대해 $g_i$를 정규분포 $N(0,\Sigma_i)$의 확률밀도함수라 하자. 그러면

\[
\int \frac{g_1g_2}{g_0} = \left[ \mbox{det} (I - \Sigma_0^{-2}(\Sigma_1- \Sigma_0)(\Sigma_2-\Sigma_0)\right]^{-1/2}
\]
이다.\\

이때 주어진 $(\gamma_{-1}, \lambda_{-1})$에 대해식  \eqref{16}의 좌변에서 적분 $ \int \left(\frac{d\overline{\mathbb{P}}_{(1,1,\gamma_{-1}, \lambda_{-1})} }{d\overline{\mathbb{P}}_{(1,0,\gamma_{-1}, \lambda_{-1})} } \right)^2 d\overline{\mathbb{P}}_{(1,0,\gamma_{-1}, \lambda_{-1})}$을 생각해본다. 
 $\overline{\mathbb{P}}_{(1,0,\gamma_{-1}, \lambda_{-1})}$는 $p$차원 정규분포 $N_p(0,\Sigma_0)$를 따르는 $n$개의 i.i.d. 다변량 정규분포들에 대한 결합분포이고 $\overline{\mathbb{P}}_{(1,1,\gamma_{-1}, \lambda_{-1})}$는 $p$차원 정규분포를 따르는 $n$개의 표본에 대한 결합분포들 $\binom{p_{\lambda_{-1}}}{k}$개의 평균이다.  이제 $\lambda_1$과 $\lambda_1'$을  $\Lambda_1(\lambda_{-1})$에서 임의로 뽑고, 이들과 대응하는 공분산 행렬들을 각각 $\Sigma_1$과 $\Sigma_2$라 하자. $g_i$ ($i=0,1,2$)를 정규분포 $N_p(0,\Sigma_i)$의 확률밀도함수라 하면 적분 $ \int \left(\frac{d\overline{\mathbb{P}}_{(1,1,\gamma_{-1}, \lambda_{-1})} }{d\overline{\mathbb{P}}_{(1,0,\gamma_{-1}, \lambda_{-1})} } \right)^2 d\overline{\mathbb{P}}_{(1,0,\gamma_{-1}, \lambda_{-1})}$은 $\left( \int \frac{g_1 g_2}{g_0}\right)^n$ 형태의 적분들의 합으로 이루어진다는 것을 알 수 있다.  그래서 $R_{\lambda_1, \lambda_1'}^{\gamma_{-1}, \lambda_{-1}}$을
 
 \[
 R_{\lambda_1, \lambda_1'}^{\gamma_{-1}, \lambda_{-1}} := -\log \mbox{det}\left\{I_p - \Sigma_0^{-2}(\Sigma_0-\Sigma_1)(\Sigma_0-\Sigma_2) \right\}
 \]
와 같이 쓴다면, 식 \eqref{16}은 다음과 같이 쓸 수 있다.

\begin{equation}\label{18}\tag{18}
\begin{aligned}
&\mathbb{E}_{(\gamma_{-1}, \lambda_{-1})}\left\{ \int \left(\frac{d\overline{\mathbb{P}}_{(1,1,\gamma_{-1}, \lambda_{-1})} }{d\overline{\mathbb{P}}_{(1,0,\gamma_{-1}, \lambda_{-1})} } \right)^2 d\overline{\mathbb{P}}_{(1,0,\gamma_{-1}, \lambda_{-1})} -1 \right\}  \\
&= \mathbb{E}_{(\gamma_{-1}, \lambda_{-1})} \left[ \mathbb{E}_{(\lambda_1, \lambda_1')|\lambda_{-1}} \left\{\exp\left( \frac n2 R_{\lambda_1, \lambda_1'}^{\gamma_{-1}, \lambda_{-1}}\right) \right\}\right]\\
&= \mathbb{E}_{(\lambda_1, \lambda_1')}   \left[ \mathbb{E}_{(\gamma_{-1}, \lambda_{-1})|(\lambda_1, \lambda_1')}\left\{\exp\left( \frac n2 R_{\lambda_1, \lambda_1'}^{\gamma_{-1}, \lambda_{-1}}\right) \right\}\right].
\end{aligned}
\end{equation}
이때 $\lambda_1, \lambda_1' |\lambda_{-1} \overset{i.i.d}{\sim}\mbox{Unif}\{\Lambda_1(\lambda_{-1})\}$, $(\gamma_{-1}, \lambda_{-1})|(\lambda_1, \lambda_1') \sim \mbox{Unif}\{ \Theta_{-1}(\lambda_1, \lambda_1')\}$와 같은 분포가 주어져 있다고 생각하며, $\Theta_{-1}(\cdot, \cdot)$은

\[
\Theta_{-1}(a_1, a_2) := \{0,1\}^{r-1} \times \{c \in \Lambda_{-1} \ : \ \exists \theta_i \in \Theta, \ i=1,2, \ \mbox{ s.t. } \ \lambda_1(\theta_i)= a_i, \ \lambda_{-1}(\theta_i) = c \}
\]
와 같이 주어져있다. Lemma 6의 결과를 이용하면 식 \eqref{18}은 아래의 식에 의해 유계이다:

\begin{equation}\label{20}\tag{20}
\begin{aligned}
\mathbb{E}_J &\left[\exp \left\{-n \log (1 -J\epsilon_{np}^2) \right\}\mathbb{E}_{(\lambda_1, \lambda_1')|J} \left\{ \mathbb{E}_{(\gamma_{-1}, \lambda_{-1})|(\lambda_1, \lambda_1')}\exp\left(\frac n2 R_{1,\lambda_1,\lambda_1'}^{\gamma_{-1}, \lambda_{-1}} \right)\right\} \right]\\
&\le \mathbb{E}_J\left[\exp\left\{ -n\log (1- J\epsilon_{np}^2)\right\}\frac32 -1 \right].
\end{aligned}
\end{equation}
여기서 $J$는 $\lambda_1$과 $\lambda_1'$에서 0이 아닌 성분이 같은 위치에 몇개나 있는지를 센 숫자를 뜻한다. 즉, $J = \lambda_1^T \lambda_1'$이다. $\lambda_1, \lambda_1' |\lambda_{-1} \overset{i.i.d}{\sim}\mbox{Unif}\{\Lambda_1(\lambda_{-1})\}$임을 이용하면 각 $j=0, \dots, k$마다
\[
\mathbb{E}_J \{ I(J=j) | \lambda_{-1} \} = \frac{\binom{k}{j} \binom{p_{\lambda_{-1}}-k }{k-j}}{\binom{p_{\lambda_{-1}}}{k}} = \left(\frac{k!}{(k-j)!}\right)^2 \frac{\{(p_{\lambda_{-1}}-k)! \}^2}{p_{\lambda_{-1}}! (p_{\lambda_{-1}} -2k +j)!} \frac{1}{j!} \le \left( \frac{k^2}{p_{\lambda_{-1}} -k}\right)^j
\]
임을 얻게 된다. 그러므로 모든 $\lambda_{-1}$에 대해 $p_{\lambda_{-1}} \ge p/4-1$가 성립한다는 것을 이용하면
\[
\mathbb{E}_J I (J=j) = \mathbb{E}_{\lambda_{-1}} \left[\mathbb{E}_J \left\{ I(J=j) | \lambda_{-1}\right\}\right] \le \mathbb{E}_{\lambda_{-1}}\left\{ \left( \frac{k^2}{p_{\lambda_{-1}} -k}\right)^j\right\} \le \left( \frac{k^2}{p/4-1-k}\right)^j
\]
를 얻는다. 따라서, 식 \eqref{20}은 $p$가 충분히 크다면 다음 식에 의해 유계이다:

\[\begin{aligned}
\sum_{j=0}^k &\left(\frac{k^2}{p/4-1-k} \right)^j \left[\exp\left\{ -n\log(1-j \epsilon_{np}^2)\right\}\frac 32 -1 \right]\\
&= \frac12 + \sum_{j=1}^k \left(\frac{k^2}{p/4-1-k} \right)^j \left[\exp\left\{ -n\log(1-j \epsilon_{np}^2)\right\}\frac 32 -1 \right]\\
&\le \frac12 + \frac32 \left(\frac{k^2}{p/4-1-k} \right)^j  p^{2\nu^2 j} \le  \frac12 + \frac{3C}{2} p^{-\epsilon j}p^{(\epsilon/2) j} \le c_2^2,
\end{aligned}\]
이때 $C>0$는 상수이고 $c_2^2 = 3/4 <1$이며, 두번째 부등식은 $s_0^2 = O(p^{3-\epsilon})$과 $\nu = \sqrt{\epsilon/4}$임을 이용하면 얻을 수 있다. 이로써 (13)식에 대한 증명이 마무리되었다.


% 김성민

\section{백승찬 - Lemma 6}

% 컴파일 에러 - 수정 필요
% 

\textbf{Lemma6}
    
    \vspace{5mm}
    
    
    \noindent If 
    $  s_0^2(\log{p})^3 = O(p^2n)$ 
    and 
    $s_0^2 = O(p^{3-\epsilon})$ for some $\epsilon>0,$ 
    then
    $${R^{\gamma_{-1},\lambda_{-1}}_{\lambda_1,\lambda_1^'}}
    = -2\log(1-J\epsilon^{2}_{np}) +  R^{\gamma_{-1},\lambda_{-1}}_{1,\lambda_1,\lambda_1^'},$$
    where ${R^{\gamma_{-1},\lambda_{-1}}_{\lambda_1,\lambda_1^'}} = -\log \det(I_p-\Sigma^{-2}(\Sigma_0-\Sigma_1)(\Sigma_0-\Sigma_2))$
    and $R^{\gamma_{-1},\lambda_{-1}}_{1,\lambda_1,\lambda_1^'}$ 
    satisfies
    $$\bbE_{{(\lambda_{1},\lambda_1^{'})}\vert J}\left[\bbE_{(\gamma_{-1},\lambda_{-1}) \vert (\lambda_1,\lambda_1^{'})} {\left(\exp({\cfrac{n}{2}} {R^{\gamma_{-1},\lambda_{-1}}_{1,\lambda_1,\lambda_1^'}})\right)}\right]
    \leq {\cfrac{3}{2}}$$
    

\vspace{5mm}

\begin{proof}

\begin{itemize}
    \item 먼저 용어에 대한 정의를 한다.
    
    
\begin{itemize}
    \item $r=\left\lfloor{\cfrac{p}{2}} \right\rfloor,~ \epsilon_{np} = \nu{\sqrt{\cfrac{\log{p}}{n}}},~  \nu = \sqrt{\cfrac{\epsilon}{4}}$
    
    \item Define parameter space   :
    $$B_1  :=   \left\{
    \Sigma(\theta):\Sigma(\theta)  =I_{p} + 
    \epsilon_{np}\displaystyle\sum_{m=0}^{r}{{\gamma_m}{A_{m}(\lambda_m)},   \theta = (\gamma , \lambda)\in \btheta}\right\}$$
    
    %$$\left(  \left\{ \dfrac{1}{2} \right\} \right)$$
    
    \item 
    
    $\Lambda :=\Big\{ \lambda=(\lambda_1,\ldots,\lambda_r)^{\top}:\lambda_{m}
    =(\lambda_{mi})\in \{0,1\}^{p}  ,   \Vert {\lambda_m}\Vert_{0}=k 
      ,   \displaystyle\sum\limits_{i=1}^{p-r}{\lambda_{mi}}=0, m\in{\{1,\ldots,r\}},
    ~~~\text{satisfying}~~~
    \max\limits_{1\leq{i}\leq{p}}\displaystyle\sum_{m=1}^{r}
    \lambda_{mi} \leq {2k}, k=\lceil{c_{np}/2}\rceil -1,
    c_{np} = \lceil{s_{0}/p}\rceil \Big\}$
\end{itemize}

\end{itemize}





%\newpage

\vspace{5mm}

%이제 ${R^{\gamma_{-1},\lambda_{-1}}_{\lambda_1,\lambda_1^'}}$을 건드려보려고 한다.

\begin{itemize}
    \item Consider ${R^{\gamma_{-1},\lambda_{-1}}_{\lambda_1,\lambda_1^'}}$, then
    
    
    

    \begin{itemize}
    \item 먼저 A를 정의한다  :  \\
    $$A=[I-(\Sigma_{0}-\Sigma_{1})(\Sigma_{0}-\Sigma_{2})]^{-1}
    (\Sigma_{0}^{-2} - I)(\Sigma_0 -\Sigma_1)(\Sigma_0 -\Sigma_2)$$
    
    \item ${R^{\gamma_{-1},\lambda_{-1}}_{1,\lambda_1,\lambda_1^'}} \overset{\underset{\mathrm{def}}{}}{=} -\log\det(I-A)$
    
    \item ${R^{\gamma_{-1},\lambda_{-1}}_{\lambda_1,\lambda_1^'}} \overset{\underset{\mathrm{def}}{}}{=}
    -\log\det[I-(\Sigma_0 -\Sigma_1)(\Sigma_0 -\Sigma_2)-(\Sigma_{0}^{-2}-I)(\Sigma_0 -\Sigma_1)(\Sigma_0 -\Sigma_2)]$
    

    
    \end{itemize}

\end{itemize}

    \vspace{15mm}

\begin{itemize}
    \item 이제 ${R^{\gamma_{-1},\lambda_{-1}}_{\lambda_1,\lambda_1^'}}$에 대해서 다시 살펴보게 되면
    
    
\begin{itemize}
    
    \item Note that
    $$\begin{aligned}
    {R^{\gamma_{-1},\lambda_{-1}}_{\lambda_1,\lambda_1^'}} 
    & = -\log\det[I-(\Sigma_0 -\Sigma_1)(\Sigma_0 -\Sigma_2)-(\Sigma_{0}^{-2}-I)(\Sigma_0 -\Sigma_1)(\Sigma_0 -\Sigma_2)]\\
    & = -\log\det[\{I-(\Sigma_0 -\Sigma_1)(\Sigma_0 -\Sigma_2)\}\{(I-A)\}]\\
    & = -\log\det[I-(\Sigma_0 -\Sigma_1)(\Sigma_0 -\Sigma_2)]-\log\det[I-A]\\
    & = -\log\det[I-(\Sigma_0 -\Sigma_1)(\Sigma_0 -\Sigma_2)]+{R^{\gamma_{-1},\lambda_{-1}}_{1,\lambda_1,\lambda_1^'}}
    \end{aligned}$$
    
    
    
    
    \item Note that $$(I-A)=I-[I-(\Sigma_{0}-\Sigma_{1})(\Sigma_{0}-\Sigma_{2})]^{-1}$$
    위 식의 양변에 $[I-(\Sigma_{0}-\Sigma_{1})(\Sigma_{0}-\Sigma_{2})]$ 를 곱한 다음 양변에 $-\log\det$ 를 취해주게되면
    $$\begin{aligned}
        \Rightarrow -\log & \det ([I-(\Sigma_{0}-\Sigma_{1})(\Sigma_{0}-\Sigma_{2})](I-A))\\
        &=-\log\det(I-(\Sigma_{0}-\Sigma_{1})(\Sigma_{0}-\Sigma_{2})-(\Sigma_{0}^{-2} - I)(\Sigma_0 -\Sigma_1)(\Sigma_0 -\Sigma_2))
    \end{aligned}$$
    
    
    
    
\end{itemize}

    
\end{itemize}




%$$\begin{aligned}
%\text{이제}  -\log\det(I-(\Sigma_{0}-\Sigma_{1})(\Sigma_{0}-\Sigma_{2}) = -2\log(1-J\epsilon_{np}^{2}) \text{임을 보이면 된다.}
%\text{그러기에 앞서} \Sigma_{0} , \Sigma_{1}, \Sigma_{2} \text{에 대한 설명을 하고자 한다.}
%\end{aligned}$$


\vspace{20mm}

\begin{itemize}
    
    \item 이제 다음과 같은 사실을 보이려고 한다.
    $$- \log \det ( I - (\Sigma_0 - \Sigma_1 ))  (\Sigma_0 - \Sigma_2) = - 2 \log (1 - J \epsilon_{np}^2) $$

\begin{itemize}
    
    \item 이때 $\Sigma_{0},\Sigma_{1},\Sigma_{2}$ 의 정의에 관해 설명하고자 한다.
    

   
   \begin{itemize}
    \item[(1)]
$$\Sigma_{0} =     
\begin{pmatrix} 
1 & \textbf{0}_{1\times(p-1)} \\
\textbf{0}_{(p-1)\times1} & \textbf{S}_{(p-1)\times(p-1)}
\end{pmatrix}$$

\vspace{5mm}

\item[(2)]
$$\Sigma_{1} =     
\begin{pmatrix} 
1 & \textbf{v}_{1\times(p-1)} \\
\textbf{v}_{(p-1)\times1} & \textbf{S}_{(p-1)\times(p-1)}
\end{pmatrix}$$

\vspace{5mm}

\item[(3)]
$$\Sigma_{2} =     
\begin{pmatrix} 
1 & {\textbf{v}}^*_{1\times(p-1)} \\
{\textbf{v}}^*_{(p-1)\times1} & \textbf{S}_{(p-1)\times(p-1)}
\end{pmatrix}$$


\newpage

    \item[(4)]
$$\Sigma_{1} - \Sigma_{0} =     
\begin{pmatrix} 
0 & \textbf{v}_{1\times(p-1)} \\
\textbf{v}_{(p-1)\times1} & \textbf{0}_{(p-1)\times(p-1)}
\end{pmatrix}$$

\vspace{5mm}

\item[(5)]
$$\Sigma_{2} - \Sigma_{0} =     
\begin{pmatrix} 
0 & {\textbf{v}}^*_{1\times(p-1)} \\
{\textbf{v}}^*_{(p-1)\times1} & \textbf{0}_{(p-1)\times(p-1)}
\end{pmatrix}$$

\vspace{5mm}

\item[(6)]
$$\textbf{v}_{1\times(p-1)}=(v_j)_{2\leq{j}\leq{p}}= \begin{cases} 0 & (~{2\leq{j}\leq{p-r}}) \\ 0~or~\epsilon_{np} & (~ {{p-r+1}\leq{j}\leq{p}}) \end{cases} with~~ \|\textbf{v}\|_0 = k$$

\vspace{5mm}

\item[(7)]
$${\textbf{v}}^*_{1\times(p-1)}=({v}^*_j)_{2\leq{j}\leq{p}}= \begin{cases} 0 & (~{2\leq{j}\leq{p-r}}) \\ 0~or~\epsilon_{np} & (~ {{p-r+1}\leq{j}\leq{p}}) \end{cases} with~~ \|{\textbf{v}}^*\|_0 = k$$

\vspace{5mm}

\item[(8)]
$$(v_j)= \begin{cases} \epsilon_{np} & (~{p-r+1\leq{j}\leq{p-r+k}}) \\ 0 & (~ {o.w.}) \end{cases}$$

\vspace{5mm}

\item[(9)]
$$(v_j)^*= \begin{cases} \epsilon_{np} & (~{p-r+k-J+1\leq{j}\leq{p-r+2k-J}}) \\ 0 & (~ {o.w.}) \end{cases}$$


\vspace{5mm}

\item[(10)]
$$\textbf{S}_{(p-1)\times(p-1)}=({s}_{ij})_{2\leq{i,j}\leq{p}}~\text{is~uniquely~determined~by~}(\gamma_{-1},\lambda_{-1}),$$

 $$\text{where~}(\gamma_{-1},\lambda_{-1})=((\gamma_2,\ldots,\gamma_r),(\lambda_2,\ldots,\lambda_r)),\text{and}~({s}_{ij})=\begin{cases} 1 & (~i=j) \\ \epsilon_{np} & (~\gamma_{i}={\lambda_i}(j)=1) \\ 0 & (~o.w.) \end{cases}$$
 
\vspace{5mm}

\end{itemize}
    
\newpage

    \item 위의 정의를 바탕으로 $I-(\Sigma_0 -\Sigma_1)(\Sigma_0 -\Sigma_2)$ 의 특징과 그 성질을 알아보고자 한다.
    \vspace{5mm}
    \item Define J to be # of overlapping $\epsilon_{np}$'s between $\Sigma_1$ and $\Sigma_2$ on the 1st row, $Q \triangleq (q_{ij})_{1\leq{i,j}\leq{p}} = (\Sigma_0 -\Sigma_1)(\Sigma_0 -\Sigma_2)$
    \vspace{5mm}
    \item Let index subset $I_r$ and $I_c$ in ${2,\ldots,p}$ with $\text{Card}(I_r)=\text{Card}(I_c)=k$ , and  $\text{Card}(I_r\cap{I_c})=J$, s.t. $(q_{ij})=\begin{cases} J{\epsilon^2_{np}} & (~i=j=1) \\ {\epsilon^2_{np}} & (~i\in I_r \And j\in I_c) \\ 0 & (~o.w.) \end{cases}$
    \vspace{5mm}
    \item $Q$가 어떤 원리로 구성되는지 살펴보기 위해 간단한 예시를 들어보고자 한다.
     $$(\Sigma_0 -\Sigma_1)=(\Sigma_0 -\Sigma_2)=
     \begin{pmatrix} 
0 & 0 & \epsilon_{np} & \epsilon_{np} & 0\\
0 & 0 & 0 & 0 & 0\\
\epsilon_{np} & 0 & 0 & 0 & 0\\
\epsilon_{np} & 0 & 0 & 0 & 0\\
0 & 0 & 0 & 0 & 0
\end{pmatrix}$$
    \item 이때 $(\Sigma_0 -\Sigma_1)(\Sigma_0 -\Sigma_2)$ 은
    $$(\Sigma_0 -\Sigma_1)(\Sigma_0 -\Sigma_2)=
         \begin{pmatrix} 
2{\epsilon^2_{np}} & 0 & 0 & 0 & 0\\
0 & 0 & 0 & 0 & 0\\
0 & 0 & {\epsilon^2_{np}} & {\epsilon^2_{np}} & 0\\
0 & 0 & {\epsilon^2_{np}} & {\epsilon^2_{np}} & 0\\
0 & 0 & 0 & 0 & 0
\end{pmatrix}$$
와 같이 구성되는데 이때 $(1,1)$성분에서 $2$가 의미하는 것은 위에서 정의한 $J$와 같음을 알 수 있다. 즉 $Q$의 대각성분중 0이 아닌 것의
갯수와 일치하는 것이다. 또한 모든 가능한 $(\Sigma_0 -\Sigma_1)(\Sigma_0 -\Sigma_2)$에 대하여 linearly independent vector는 단 2개밖에
존재하지 않으므로 $rank[(\Sigma_0 -\Sigma_1)(\Sigma_0 -\Sigma_2)]=2$ 임을 알 수 있다.
    \vspace{5mm}
       \item 이러한 사실들을 토대로 $I-(\Sigma_0 -\Sigma_1)(\Sigma_0 -\Sigma_2)$ 의 characteristic polynomial을 구해보고자 한다. 즉
    $det[\lambda{I}-(\Sigma_0 -\Sigma_1)(\Sigma_0 -\Sigma_2)]$을 구하는 것인데 일반적인 경우의 값을 구하기 쉽지않아 특수한 경우에 한해서
    구해보려고 한다. 즉 $J=k$인 경우만을 고려해보고자 한다. 이러한 경우
    
    
$$
    \det[\lambda{I}-(\Sigma_0 -\Sigma_1)(\Sigma_0 -\Sigma_2)]
    & = \det\left[\begin{pmatrix} 
{\lambda I_{p-k} -     
\begin{pmatrix} 
J{\epsilon^2_{np}} & \textbf{0} \\
\textbf{0}^{\top} & \textbf{O}
\end{pmatrix}} & \textbf{0}_{(p-k)\times(k)} \\
\textbf{0}_{(k)\times(p-k)} & \lambda I_k -{\epsilon^2_{np}}{\textbf{1}_k}{\textbf{1}^{\top}_k}
\end{pmatrix} \right]
$$


$$=\det \left[{\lambda I_{p-k} -     
\begin{pmatrix} 
J{\epsilon^2_{np}} & \textbf{0} \\
\textbf{0}^{\top} & \textbf{O}
\end{pmatrix}} \right] \det \left[\lambda I_k -{\epsilon^2_{np}}{\textbf{1}_k}{\textbf{1}^{\top}_k} \right]$$
    \vspace{5mm}$$

\newpage
$$
=(\lambda - J{\epsilon^2_{np}})\lambda ^{p-k-1} \det \left[\lambda I_k -{\epsilon^2_{np}}{\textbf{1}_k}{\textbf{1}^{\top}_k} \right]
$$

$$
=(\lambda - J{\epsilon^2_{np}})\lambda ^{p-k-1} \det \left(1 -{\epsilon^2_{np}}{\textbf{1}^{\top}_k}(\lambda I_k)^{-1}{\textbf{1}_k} \right) \det \left(\lambda I_k)\right)
$$   

$$
=(\lambda - J{\epsilon^2_{np}})\lambda ^{p-k-1} \left(1 -  \cfrac{J}{\lambda}{\epsilon^2_{np}} \right)\lambda^k
$$  

$$
=(\lambda - J{\epsilon^2_{np}})^{2}\lambda ^{p-2}
$$ 
    
\end{itemize}

\begin{itemize}
    
    \item Note that
    $$\begin{aligned}
    {R^{\gamma_{-1},\lambda_{-1}}_{\lambda_1,\lambda_1^'}} 
    & = -\log\det[I-(\Sigma_0 -\Sigma_1)(\Sigma_0 -\Sigma_2)]+{R^{\gamma_{-1},\lambda_{-1}}_{1,\lambda_1,\lambda_1^'}}\\
    & = -\log\left(1-J{\epsilon^2_{np}} \right)^2 +{R^{\gamma_{-1},\lambda_{-1}}_{1,\lambda_1,\lambda_1^'}} \\
    & = -2\log\left(1-J{\epsilon_{np}} \right) +{R^{\gamma_{-1},\lambda_{-1}}_{1,\lambda_1,\lambda_1^'}} \\
    \end{aligned}$$
    
    
    
    
    \item Note that $$ \det[\lambda{I}-(\Sigma_0 -\Sigma_1)(\Sigma_0 -\Sigma_2)]
    \overset{\lambda = 1}{=}\left(1-J{\epsilon^2_{np}} \right)^2$$
    
    
    
\end{itemize}



\end{itemize}

\newpage
\begin{itemize}
    
    \item 이제 다음과 같은 사실을 보이려고 한다.
        $$\bbE_{{(\lambda_{1},\lambda_1^{'})}\vert J}\left[\bbE_{(\gamma_{-1},\lambda_{-1}) \vert (\lambda_1,\lambda_1^{'})} {\left(\exp({\cfrac{n}{2}} {R^{\gamma_{-1},\lambda_{-1}}_{1,\lambda_1,\lambda_1^'}})\right)}\right]
    \leq {\cfrac{3}{2}}$$
    \begin{itemize}
    
    \item Recall :
    $A=[I-(\Sigma_{0}-\Sigma_{1})(\Sigma_{0}-\Sigma_{2})]^{-1}
    (\Sigma_{0}^{-2} - I)(\Sigma_0 -\Sigma_1)(\Sigma_0 -\Sigma_2)$
    \vspace{5mm}
        \item It is important to observe that $rank(A)\leq2$ due to the structure of 
        $(\Sigma_{0}-\Sigma_{1})(\Sigma_{0}-\Sigma_{2})$. Let $\varrho$ be an eigenvalue of $A$. It is easy to see that $|{\varrho}|\leq\norm{A}$.
        \vspace{5mm}
    \item We wish to find the upper bound for$\norm{A}$. To proceed, first we can see that
    $$
    \norm{A}\leq\norm{[I-(\Sigma_{0}-\Sigma_{1})(\Sigma_{0}-\Sigma_{2})]^{-1}}\norm{\Sigma_{0}^{-2} - I}\norm{(\Sigma_0 -\Sigma_1)(\Sigma_0 -\Sigma_2)}.
    $$
    \vspace{5mm}
    \item From Tony Cai's "Optimal rates of convergence for sparse covariance matrix estimation" (22), we can see that
    $$
    \norm{\Sigma_{1}-\Sigma_{0}}\leq{\norm{\Sigma_{1}-\Sigma_{0}}}_1 = k\epsilon_{np}\leq 2k\epsilon_{np}\leq c_{np}{\epsilon}^{1-q}_{np}\leq Mv^{1-q}<\dfrac{1}{3}
    $$
    Similarly, we can see that for $\norm{I-\Sigma_0}$ and $\norm{(\Sigma_{0}-\Sigma_{1})(\Sigma_{0}-\Sigma_{2})}$,
    $$
    \norm{I-\Sigma_0} \leq {\norm{I-\Sigma_0}}_1 = k\epsilon_{np} < \dfrac{1}{3},~
    \norm{(\Sigma_{0}-\Sigma_{1})(\Sigma_{0}-\Sigma_{2})}\leq \dfrac{1}{3}\times \dfrac{1}{3} < 1. 
    $$
    \vspace{5mm}
    \item Note that : $|\log(1-x)|\leq2|x|,~\text{for}~|x|<\dfrac{1}{6}$,~~ ${R^{\gamma_{-1},\lambda_{-1}}_{1,\lambda_1,\lambda_1^'}} \overset{\underset{\mathrm{def}}{}}{=} -\log\det(I-A)$
    
    
    $$\begin{aligned}
    \Rightarrow {R^{\gamma_{-1},\lambda_{-1}}_{1,\lambda_1,\lambda_1^'}}
    & = -\log\det(I-A) \leq -2(1-\det(I-A))\\
    & \leq |-2(1-\det(I-A))| \leq |-2(\det(I)-\det(I-A))\\
    & \leq 2|\det(I)-\det(I-A)| = 2m\norm{I-(I-A)}\\
    & = 2m\norm{A},~~ \text{for~some}~m>0.
    \end{aligned}$$
    
    $$
    \Rightarrow {R^{\gamma_{-1},\lambda_{-1}}_{1,\lambda_1,\lambda_1^'}}
    & \leq 4\norm{A},~~\text{for}~m=2.
    $$
    \vspace{5mm}
    \item Note that : $\exp\left(\dfrac{n}{2} {R^{\gamma_{-1},\lambda_{-1}}_{1,\lambda_1,\lambda_1^'}}  \right) \leq \exp\left(2n\norm{A}\right)$,~~
    \vspace{5mm}    
    \item Note that : for any square matrix $B$, following statement is true :
    $$\left(\displaystyle\sum_{m=1}^{\infty}{B^m}\right)^2 = \left(B + B^2 + \ldots\right)\left(B + B^2 + \ldots\right) = B^2 + 2B^3 + 3B^4 + \ldots = \displaystyle\sum_{m=0}^{\infty}{mB^{m+1}}$$
    \vspace{5mm}    
    \item From above statement, we can write the following :
    $$\begin{aligned}
    \Sigma^{-2}_0 - I
    & = \left(I-\left(I-\Sigma_0\right)\right)^{-2} - I = \left(I + \displaystyle\sum_{m=1}^{\infty}{{(I-\Sigma_0)}^m}\right)^2 - I\\
    & = I + 2\displaystyle\sum_{m=1}^{\infty}{{(I-\Sigma_0)}^m} + \left(\displaystyle\sum_{m=1}^{\infty}{{(I-\Sigma_0)}^m}\right)^2 - I\\
    & = 2\displaystyle\sum_{m=1}^{\infty}{{(I-\Sigma_0)}^m} + \left(\displaystyle\sum_{m=1}^{\infty}{{(I-\Sigma_0)}^m}\right)^2\\
    & = 2\displaystyle\sum_{m=0}^{\infty}{{(I-\Sigma_0)}^{m+1}} + \left(\displaystyle\sum_{m=1}^{\infty}{{(I-\Sigma_0)}^m}\right)^2\\
    & = 2\displaystyle\sum_{m=0}^{\infty}{{(I-\Sigma_0)}^{m+1}} + \displaystyle\sum_{m=0}^{\infty}{m{(I-\Sigma_0)}^{m+1}}\\
    & = \left[\displaystyle\sum_{m=0}^{\infty}{(m+2){(I-\Sigma_0)}^m}\right]\left(I-\Sigma_0\right).
    \end{aligned}$$
    \vspace{5mm}
    \item We can see that
    $$
    \norm{\displaystyle\sum_{m=0}^{\infty}{(m+2){(I-\Sigma_0)}^m}} = \displaystyle\sum_{m=0}^{\infty}{(m+2)\norm{I-\Sigma_0}^m} < \displaystyle\sum_{m=0}^{\infty}{(m+2)\left(\dfrac{1}{3}\right)^m}=\dfrac{13}{4}<4.
    $$
    \vspace{5mm}    
    \item Define $A_*=(I-\Sigma_0)(\Sigma_{0}-\Sigma_{1})(\Sigma_{0}-\Sigma_{2})$. Then
    $$\begin{aligned}
    \norm{A}
    & \leq \norm{[I-(\Sigma_{0}-\Sigma_{1})(\Sigma_{0}-\Sigma_{2})]^{-1}} \norm{\left[\displaystyle\sum_{m=0}^{\infty}{(m+2){(I-\Sigma_0)}^m}\right]A_*}\\
    & \leq \norm{[I-(\Sigma_{0}-\Sigma_{1})(\Sigma_{0}-\Sigma_{2})]^{-1}}\norm{\displaystyle\sum_{m=0}^{\infty}{(m+2){(I-\Sigma_0)}^m}}\norm{A_*}\\
    & < \norm{[I-(\Sigma_{0}-\Sigma_{1})(\Sigma_{0}-\Sigma_{2})]^{-1}}\cdot 4 \cdot \norm{A_*}\\
    & < 4 \cdot \cfrac{1}{1-{\cfrac{1}{3}}\cdot{\cfrac{1}{3}}}\cdot \norm{A_*} = \dfrac{9}{2}\norm{A_*} \leq \dfrac{9}{2}\max\left\{\norm{A_*}_1,\norm{A_*}_{\infty}\right\},
    \end{aligned}$$
    where $A_* = (a^*_{ij})_{1\leq{i,j}\leq{p}}$, and $\begin{cases} \norm{A_*}_1 = \max\limits_{1\leq{m}\leq{p}}\displaystyle\sum_{j=1}^{p}|a^*_{mj}|\\ \norm{A_*}_{\infty} = \max\limits_{1\leq{m}\leq{p}}\displaystyle\sum_{i=1}^{p}|a^*_{im}|\end{cases}$
    \vspace{5mm}    
    \item Summing up above results, we obtain following :
    $$
    \exp\left(\dfrac{n}{2} {R^{\gamma_{-1},\lambda_{-1}}_{1,\lambda_1,\lambda_1^'}}  \right) \leq \exp\left(2n\norm{A}\right) = \exp\left(9n\max\left\{\norm{A_*}_1,\norm{A_*}_{\infty}\right\}\right)
    $$
    which implies
    $$
    \bbE_{{(\lambda_{1},\lambda_1^{'})}\vert J}\left[\bbE_{(\gamma_{-1},\lambda_{-1}) \vert (\lambda_1,\lambda_1^{'})} {\left(\exp({\cfrac{n}{2}} {R^{\gamma_{-1},\lambda_{-1}}_{1,\lambda_1,\lambda_1^'}})\right)}\right]$$
    
    $$\leq  \bbE_{{(\lambda_{1},\lambda_1^{'})}\vert J}\left[\bbE_{(\gamma_{-1},\lambda_{-1}) \vert (\lambda_1,\lambda_1^{'})} {\left(\exp(9n\max\left\{\norm{A_*}_1,\norm{A_*}_{\infty}\right\})\right)}\right]
    $$
    
    
\newpage
    \vspace{5mm}    
    \item But infact, :
    $$
    \bbE_{{(\lambda_{1},\lambda_1^{'})}\vert J}\left[\bbE_{(\gamma_{-1},\lambda_{-1}) \vert (\lambda_1,\lambda_1^{'})} {\left( I \left\{ \max\left\{\norm{A_*}_1,\norm{A_*}_{\infty}\right\} \geq 2tk\epsilon^3_{np} \right\}  \right)}\right]
    $$
    
    $$
    = \bbP\left( \max\left\{\norm{A_*}_1,\norm{A_*}_{\infty}\right\} \geq 2tk\epsilon^3_{np} \right)
    $$

    \vspace{5mm}    
    \item So we wish to show that
    $$
    \bbP\left(\displaystyle\sum_{j=1}^{p}{|a^*_{mj}|} \geq 2tk\epsilon^3_{np} \right) \leq \left(\dfrac{k^2}{p/8 -1 -k}  \right)^t
    $$
    which implies that
    $$
    \bbP( \max\left\{\norm{A_*}_1,\norm{A_*}_{\infty}\right\} \geq 2tk\epsilon^3_{np}) \leq 2p\left(\dfrac{k^2}{p/8 -1 -k}  \right)^t
    $$
    \vspace{5mm}    
    \item For each row $m$, define $E_m = \{1,2,\ldots,r\} \backslash \{1,m\}.$ Note that for each column of $\lambda_{E_m}$, if the column sum of $\lambda_{E_m}$ is less than or equal to $2k-2$, then the other two rows can still freely take values 0 or 1 in this column, because the total sum will still not exceed $2k$. Let $n_{\lambda_{E_m}}$ be the number of columns of $\lambda_{E_m}$ with column sum at least $2k-1$, and define $p_\lambda_{E_m} = r - n_{\lambda_{E_m}}$. Without loss of generality we assume that $k\geq 3$. Since $n_{\lambda_{E_m}}\cdot(2k-2)\geq r\cdot k$, the total number of 1's in the upper triangular matrix by the construction of the parameter setm we thus have $n_{\lambda_{E_m}} \geq r \cdot \dfrac{3}{4}$, which immediately implies $p_{\lambda_{E_m}} = r -  n_{\lambda_{E_m}} \geq \dfrac{r}{4} \geq \dfrac{p}{8} - 1$. Recall that the distribution of $(\gamma_{-1},\lambda_{-1})|(\lambda_1,{\lambda^{'}_{1}})$ is uniform over $\Theta^{-1}(\lambda_1,{\lambda^{'}_{1}})$.

    \vspace{5mm}    
    \item Recall that $J$ is the overlapping nonzero entries between the 1st rows of $\Sigma_1$ and $\Sigma_2$, i.e. $J=\lambda^{\top}_{1} \lambda^{'}_{1}$. Then we can obtain the following results : 
    $$
    \bbE_J [ I_{(J=t)} | \lambda_{E_m} ] = \cfrac{{k \choose t}{{p_{\lambda_{E_m}} -k} \choose {k-t}}}{{{p_{\lambda_{E_m}} } \choose {k}}}
    = \left[ \dfrac{k!}{(k-t)!} \right]^2 \cdot \dfrac{[(p_{\lambda_{E_m}} -k)!]^2}{p_{\lambda_{E_m}}! (p_{\lambda_{E_m}} -2k +t)!}\cdot \dfrac{1}{t!}\leq \left(\dfrac{k^2}{p_{\lambda_{E_m}}-k}\right)^j
    $$
    
    
    $$
    \Rightarrow~~
    \bbE_J [ I_{(J=t)} ] = \bbE_{\lambda_{E_m}} \left[   \bbE_J \right( I_{(J=t)} | \lambda_{E_m} \left)  \right]
    \leq \bbE_{\lambda_{E_m}} \left[\left(\dfrac{k^2}{p_{\lambda_{E_m}}-k} \right)^t \right] \leq \left( \dfrac{k^2}{p/8 -1 -k} \right)^t
    $$
    Then we can obtain the following :
    $$
    \Rightarrow~~
    \bbP\left(\displaystyle\sum_{j=1}^{p}{|a^*_{mj}|} \geq 2tk\epsilon^3_{np}~ \big\vert~ {\lambda_{E_m}} \right) \leq \left(\dfrac{k^2}{p/8 -1 -k}  \right)^t    
    $$
    which implies for every $t>2$,
    $$
    \Rightarrow~~
    \bbP\left(\displaystyle\sum_{j=1}^{p}{|a^*_{mj}|} \geq 2tk\epsilon^3_{np} {\lambda_{E_m}} \right) \leq \left(\dfrac{k^2}{p/8 -1 -k}  \right)^{t-1}   
    $$

\newpage
    This implies : $    \bbP( \max\left\{\norm{A_*}_1,\norm{A_*}_{\infty}\right\} \geq 2tk\epsilon^3_{np}) \leq 2p\left(\dfrac{k^2}{p/8 -1 -k}  \right)^{t-1}$ for every $t>2$.~ so~ $    \bbE_{{(\lambda_{1},\lambda_1^{'})}\vert J}\left[\bbE_{(\gamma_{-1},\lambda_{-1}) \vert (\lambda_1,\lambda_1^{'})} {\left( I \left\{ \max\left\{\norm{A_*}_1,\norm{A_*}_{\infty}\right\} \geq 2tk\epsilon^3_{np} \right\}  \right)}\right]   
    = \bbP\left( \max\left\{\norm{A_*}_1,\norm{A_*}_{\infty}\right\} \geq 2tk\epsilon^3_{np} \right) \leq 2p\left(\dfrac{k^2}{p/8 -1 -k}  \right)^{t-1}$for every $t>2$.
    
    \vspace{5mm}    
    \item Recall that
    $$
    \bbE_{{(\lambda_{1},\lambda_1^{'})}\vert J}\left[\bbE_{(\gamma_{-1},\lambda_{-1}) \vert (\lambda_1,\lambda_1^{'})} {\left(\exp({\cfrac{n}{2}} {R^{\gamma_{-1},\lambda_{-1}}_{1,\lambda_1,\lambda_1^'}})\right)}\right]$$
    
    $$\leq  \bbE_{{(\lambda_{1},\lambda_1^{'})}\vert J}\left[\bbE_{(\gamma_{-1},\lambda_{-1}) \vert (\lambda_1,\lambda_1^{'})} {\left(\exp(9n\max\left\{\norm{A_*}_1,\norm{A_*}_{\infty}\right\})\right)}\right].
    $$

    \vspace{5mm}    
    \item Note that for any r.v. $X\geq0 \And $ constant $a\geq 0$, it is known that
    $$\begin{aligned}
    \bbE[X]
    & = \int_{x\geq0}^{} P(X>x)~ dx = \int_{x\leq a}^{} P(X>x)~ dx + \int_{x>a}^{} P(X>x)~ dx\\
    & = \int_{x\leq a}^{} (1-F(x))~ dx + \int_{x>a}^{} P(X>x)~ dx\\
    & = \Big[(1-F(x)) \Big]_{0}^{a} + \int_{0}^{a} {xf(x)}~ dx + \int_{x>a}^{} P(X>x)~ dx \\
    & = a(1-F(a)) + \int_{0}^{a} {xf(x)}~ dx + \int_{x>a}^{} P(X>x)~ dx \\
    & \leq a +  \int_{x>a}^{} P(X>x)~ dx
    \end{aligned}$$
    
    \vspace{5mm}    
    \item we can apply this fact to our objective, in other words, put 
    $a= \exp\left\{2Cnk{\epsilon^3_{np}}\dfrac{1+2\epsilon}{\epsilon} \right\},$ since $k=\lceil{c_{np}/2}\rceil -1,
    c_{np} = \lceil{s_{0}/p}\rceil, \epsilon_{np} = \nu\sqrt{\log p /n}, \nu = \sqrt{\epsilon / 4}$. Then we could achieve the upper bound for $a$ with the condition $  s_0^2(\log{p})^3 = O(p^2n)$ as following : (Here, we put $9=C,~ C>0,$ for convenience, which doesn't affect the upper bound we are looking for.) 
   
    $$\begin{aligned}
    a
    & = \exp\left\{2Cnk{\epsilon^3_{np}}\dfrac{1+2\epsilon}{\epsilon} \right\}\\
    & = \exp\left(2Cn\left\{ \left\lceil{\cfrac{\lceil{s_{0}/p}\rceil}{2}}\right\rceil -1 \right\}    \right)  \left( \cfrac{\epsilon}{4} \right) \left( \cfrac{\log p}{n}   \right)^{3/2}  \left( \cfrac{1+2\epsilon}{\epsilon} \right) \sqrt{\cfrac{\epsilon}{4}}\\
    & \leq \exp\left( Cn\left\{ \cfrac{s_o}{2p} + \cfrac{1}{2}  \right\}\left( \cfrac{\log p}{n}   \right)^{3/2} \left(\cfrac{1+2\epsilon}{4} \right)\sqrt{\epsilon} \right)\\
    & = \exp \left(\cfrac{1}{2}~C\left(\cfrac{1+2\epsilon}{4} \right)\sqrt{\epsilon} \left[  \left(\cfrac{s_o}{p} +1 \right)^{2} \cfrac{(\log p)^3}{n}  \right]^{1/2} \right)\\
    & \asymp \exp\left(\cfrac{1}{2}~C\left(\cfrac{1+2\epsilon}{4} \right)\sqrt{\epsilon}\right) \asymp e^{0} = 1\\
    & < \cfrac{3}{2}, ~~\text{for sufficiently small}~~ \epsilon > 0.
    \end{aligned}$$

\newpage
    \item Now, from our finding,
    $$\begin{aligned}
    \bbE_{{(\lambda_{1},\lambda_1^{'})}\vert J}\left[\bbE_{(\gamma_{-1},\lambda_{-1}) \vert (\lambda_1,\lambda_1^{'})} {\left(\exp(Cn\max\left\{\norm{A_*}_1,\norm{A_*}_{\infty}\right\})\right)}\right]
    \end{aligned}$$

    $$\begin{aligned}   
    \leq a + \int_{x>a}^{} \bbE_{{(\lambda_{1},\lambda_1^{'})}\vert J}\left[\bbE_{(\gamma_{-1},\lambda_{-1}) \vert (\lambda_1,\lambda_1^{'})} {\left( I \left\{ \max\left\{\norm{A_*}_1,\norm{A_*}_{\infty}\right\} \geq 2tk\epsilon^3_{np} \right\}  \right)}\right]~ dx
    \end{aligned}$$
    
    $$\begin{aligned}    
    & \leq \cfrac{3}{2} + \int_{t\geq{(1+2\epsilon)/\epsilon}}^{} 2Cnk{\epsilon^3_{np}}\exp\left(2Ctnk{\epsilon^3_{np}}\right)2p\left(\dfrac{k^2}{p/8 -1 -k}  \right)^{t-1}~ dt
    \end{aligned}$$
    
    $$\begin{aligned}    
    & \leq \cfrac{3}{2} + \int_{t\geq{(1+2\epsilon)/\epsilon}}^{} \exp\left\{ \log(2p)-(t-1)\log\left(\cfrac{p/8 -1 -k}{k^2}\right) +2C(t+1)nk{\epsilon^3_{np}} \right\}~ dt.
    \end{aligned}$$    
    Thus, we complete the proof if we show that the second term of last inequality is of order $o(1)$. Note that :

    $$\begin{aligned}    
    (t-1)\log\left(\cfrac{p/8 -1 -k}{k^2}\right)
    & \geq \left(1+ \cfrac{1}{\epsilon} \right)\log\left(\cfrac{p/8 -1 -k}{k^2}\right)\\
    & = \left(1+ \cfrac{1}{\epsilon} \right)\log\left(\cfrac{p/8 -1 -(s_0/{2p} +1/2)}{(s_0/{2p} +1/2)^2}\right)\\
    & \geq \left(1+ \cfrac{1}{\epsilon} \right)\log\left(\cfrac{p/8 -1 -(s_0/{p})}{(s_0/{p})^2}\right)+C^{'}\\
    & = \left(1+ \cfrac{1}{\epsilon} \right)\log\left(\cfrac{p^3/8 -p^3 -ps_0}{s^2_{0}}\right)+C^{'}\\
    & = \left(1+ \cfrac{1}{\epsilon} \right)\log\left(\cfrac{{p^{\epsilon}}{p^{3-\epsilon}}}{s^2_{0}}  \left(\cfrac{1}{8}-\cfrac{1}{p}-\cfrac{s_0}{p^2} \right)\right)+C^{'}\\
    & \geq \left(1+ \cfrac{1}{\epsilon} \right)\log(p^{\epsilon}) + C^{''}\\
    & = (1+\epsilon)\log(p) +C^{''},
    \end{aligned}$$   
    for any $t>(1+2\epsilon)/\epsilon$ and some constants $C^' > 0 $ and $C^{''} > 0$. The third inequality follows from the assumption $s^2_{0}=O(p^{3-\epsilon})$. Therefore, it implies that the second term of last inequality is of order $o(1)$, which gives the desired result:
    
        $$\bbE_{{(\lambda_{1},\lambda_1^{'})}\vert J}\left[\bbE_{(\gamma_{-1},\lambda_{-1}) \vert (\lambda_1,\lambda_1^{'})} {\left(\exp({\cfrac{n}{2}} {R^{\gamma_{-1},\lambda_{-1}}_{1,\lambda_1,\lambda_1^'}})\right)}\right]
    \leq {\cfrac{3}{2}}$$


    \end{itemize}









    
    
\end{itemize}






\end{proof}

\section*{참고문헌}

\bibliography{refs}

\end{document}